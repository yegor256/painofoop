% (The MIT License)
%
% Copyright (c) 2023-2024 Yegor Bugayenko
%
% Permission is hereby granted, free of charge, to any person obtaining a copy
% of this software and associated documentation files (the 'Software'), to deal
% in the Software without restriction, including without limitation the rights
% to use, copy, modify, merge, publish, distribute, sublicense, and/or sell
% copies of the Software, and to permit persons to whom the Software is
% furnished to do so, subject to the following conditions:
%
% The above copyright notice and this permission notice shall be included in all
% copies or substantial portions of the Software.
%
% THE SOFTWARE IS PROVIDED 'AS IS', WITHOUT WARRANTY OF ANY KIND, EXPRESS OR
% IMPLIED, INCLUDING BUT NOT LIMITED TO THE WARRANTIES OF MERCHANTABILITY,
% FITNESS FOR A PARTICULAR PURPOSE AND NONINFRINGEMENT. IN NO EVENT SHALL THE
% AUTHORS OR COPYRIGHT HOLDERS BE LIABLE FOR ANY CLAIM, DAMAGES OR OTHER
% LIABILITY, WHETHER IN AN ACTION OF CONTRACT, TORT OR OTHERWISE, ARISING FROM,
% OUT OF OR IN CONNECTION WITH THE SOFTWARE OR THE USE OR OTHER DEALINGS IN THE
% SOFTWARE.

\documentclass{article}
\usepackage{../painofoop}
\newcommand*\thetitle{Reflection}
\newcommand*\thesubtitle{Classpath, Casting, Annotations}
\begin{document}

\plush{\poopTitlePage{7}{vYw_GcDkPQA}}

\pptToc

\plush{\pptChapter[Casting]{Type Casting and Subsumption}}

\pptSection[Iterable]{Iterable \(\to\) Collection}
\begin{pptWide}{2}
Downcasting (\textcolor{red}{wrong!}):\par
{\small\begin{ffcode}
int sizeOf(Iterable items) {
  int size = 0;
  if (items (*@\textcolor{orange}{instanceof}@*) Collection) {
    size = (*@\textcolor{orange}{((Collection)}@*) items).size();
  } else {
    for (Object item : items) {
      ++size;
    }
  }
  return size;
}
\end{ffcode}
}
\par\columnbreak\par
Method overloading (\textcolor{green}{right!}):\par
{\small\begin{ffcode}
int sizeOf(Iterable items) {
  int size = 0;
  for (Object item : items) {
    ++size;
  }
  return size;
}

int sizeOf((*@\textcolor{orange}{Collection}@*) items) {
  return items.size();
}
\end{ffcode}
}
\end{pptWide}
\par
\plush{}

\pptSection[Coupling]{Implicit Coupling}
\begin{pptWide}{2}
{\small\begin{ffcode}
int sizeOf(Iterable items) {
  int size = 0;
  (*@\textcolor{orange}{if}@*) (items instanceof Collection) {
    size = ((Collection) items).size();
  } (*@\textcolor{orange}{else}@*) {
    for (Object item : items) {
      ++size;
    }
  }
  return size;
}
\end{ffcode}
}
\par\columnbreak\par
{\small\begin{ffcode}
Iterable<Book> books1 =
  new PgBooks1("localhost:5432");
int s1 = (*@\textcolor{orange}{sizeOf}@*)(books1);

Collection<Book> books2 =
  new PgBooks2("localhost:5432");
int s2 = (*@\textcolor{orange}{sizeOf}@*)(books2);
\end{ffcode}
}
\end{pptWide}
\par
It may be hard to understand why |s2| is evaluated much faster than |s1|, since the signature of |sizeOf()| is the same in both cases.
\plush{}

\pptSection[JEP-394]{Pattern Matching in Java 16}
\begin{pptWide}{2}
Java 11 (\textcolor{red}{wrong!}):
\par
{\small\begin{ffcode}
int sizeOf(Iterable items) {
  int size = 0;
  if (items (*@\textcolor{orange}{instanceof}@*) Collection) {
    size = ((*@\textcolor{orange}{(Collection)}@*) items).size();
  } else {
    for (Object item : items) {
      ++size;
    }
  }
  return size;
}
\end{ffcode}
}
\par\columnbreak\par
Java 16 (\textcolor{red}{\textbf{even worse!}}):
\par
{\small\begin{ffcode}
int sizeOf(Iterable items) {
  int size = 0;
  if ((*@\textcolor{orange}{items instanceof Collection c}@*)) {
    size = c.size();
  } else {
    for (Object item : items) {
      ++size;
    }
  }
  return size;
}
\end{ffcode}
}
\end{pptWide}
\par
\plush{}

\pptSection[Matching]{C\#, Rust, and pattern matching}
\begin{pptWide}{2}
C\#:\par
{\scriptsize\begin{ffcode}
public int sizeOf<T>(IEnumerable<T> items) {
  if (items (*@\textcolor{orange}{is}@*) IList<T> list) {
    return list.Count;
  } else {
    return // count them one by one
  }
}
\end{ffcode}
}
\par
Some other languages have \ul{pattern matching} feature,
including Kotlin, Scala, Haskell, Elixir, Swift, F\#, and Erlang, ...
which \textcolor{red}{contradicts} the principle of encapsulation.
\par\columnbreak\par
Rust:
\par
{\scriptsize\begin{ffcode}
enum Color {
  RGB(u8, u8, u8),
  Transparent
}
fn paint(c: Color) {
  (*@\textcolor{orange}{match}@*) c {
    Color::RGB(r, g, b) =>
      println!("#{r}{g}{b}"),
    Color::Transparent =>
      println!("none")
  }
}
fn main() {
  let c = Color::RGB(64, 16, 0);
  paint(c);
}
\end{ffcode}
}
\end{pptWide}
\par
\plush{}

\plush{\pptChapter[Factory]{Factory Method}}

\pptSection[IF]{Conditional object construction}
\begin{pptWide}{2}
This is \textcolor{red}{wrong}:
\par
{\scriptsize\begin{ffcode}
interface Figure
  int surface();
class Square implements Figure
class Triangle implements Figure
class Polygon implements Figure

class FactoryOfFigures
  Figure make(int sides) {
    if (sides == 3) {
      return (*@\textcolor{orange}{new Triangle}@*)();
    } else if (sides == 4) {
      return (*@\textcolor{orange}{new Square}@*)();
    } else {
      return (*@\textcolor{orange}{new Polygon}@*)(sides);
    }
  }
\end{ffcode}
}
\par\columnbreak\par
This is \textcolor{green}{better}:
\par
{\scriptsize\begin{ffcode}
class PolymorphicFigure implements Figure
  PolymorphicFigure(int sides)
  @Override int surface() {
    if (sides == 3) {
      return (*@\textcolor{orange}{new Triangle().surface()}@*);
    } else if (sides == 4) {
      return (*@\textcolor{orange}{new Square().surface()}@*);
    } else {
      return (*@\textcolor{orange}{new Polygon(sides).surface()}@*);
    }
  }
\end{ffcode}
}
\par
Here, the semantic of object construction is not visible to the client --- the \ul{coupling} is ``loose.''
\end{pptWide}
\plush{}

\pptSection[forName]{Generating class name from a string}
\begin{pptWide}{2}
This is \textcolor{red}{wrong}:
\par
{\scriptsize\begin{ffcode}
interface Figure
  int surface();

class Square implements Figure
class Triangle implements Figure
class Polygon implements Figure

class FactoryOfFigures
  Figure make(String name) throws Exception {
    Class<?> c = Class.forName(name);
    return c.getConstructor().newInstance();
  }
\end{ffcode}
}
\par\columnbreak\par
This is \textcolor{green}{better}:
\par
{\scriptsize\begin{ffcode}
class PolymorphicFigure implements Figure
  PolymorphicFigure(String name)
  @Override int surface() {
    if (name.equals("Triangle")) {
      return new Triangle().surface();
    } else if (name.equals("Square")) {
      return new Square().surface();
    } else {
      return new Polygon(?).surface();
    }
  }
\end{ffcode}
}
\par
This is better since the mechanics of class finding is explicit --- no surprises expected.
\end{pptWide}
\plush{}

\plush{\pptChapter[Classpath]{Classpath Scanning}}

\pptSection[Class]{Finding Java classes}
\begin{pptWide}{2}
{\small\begin{ffcode}
interface Foo {}

class Bar implements Foo {}

Reflections rts =
  new Reflections("");
Set<Class<?>> types = rts.get(
  SubTypes.of(Foo.class).asClass()
);
\end{ffcode}
}
\par\columnbreak\par
{\small\begin{ffcode}
public @interface Foo {}

@Foo
class Bar {}

Reflections rts =
  new Reflections("");
Set<Class<?>> types = rts.get(
  SubTypes.of(
    TypesAnnotated.with(Foo.class)
  ).asClass()
);
\end{ffcode}
}
\end{pptWide}
\par
The library is called \href{https://github.com/ronmamo/reflections}{Reflections}.
Instead, use explicit object instantiation.
\plush{}

\plush{\pptChapter{Annotations}}

\pptSection[Class]{I lieu of static methods}
\begin{pptWide}{2}
{\scriptsize\begin{ffcode}
interface Pub
  String isbn();

class Book implements Pub
  @Override public String isbn()
    /* ... */
  public static String category()
    return "book";

class Journal implements Pub
  @Override public String isbn()
    /* ... */
  public static String category()
    return "journal";
\end{ffcode}
}
\par\columnbreak\par
{\scriptsize\begin{ffcode}
interface Pub
  String isbn();

@Target(ElementType.CLASS)
@Retention(RetentionPolicy.SOURCE)
public @interface Category
  String value();

@Category("book")
class Book implements Pub
  @Override public String isbn()
    /* ... */

@Category("journal")
class Journal implements Pub
  @Override public String isbn()
    /* ... */
\end{ffcode}
}
\end{pptWide}
\par
\plush{}

\pptSection[Method]{Locating methods}
\begin{pptWide}{2}
{\small\begin{ffcode}
@Target(ElementType.METHOD)
@Retention(RetentionPolicy.SOURCE)
public @interface Path
  String url;

class BookController
  @Path("/book-title")
  String title()
    // Build HTML page and return it
\end{ffcode}
}
\par\columnbreak\par
{\scriptsize\begin{ffcode}
String dispatch(String url) {
  c = new BooksController();
  b = BooksController.class;
  for (Method m : b.getDeclaredMethods()) {
    if (m.isAnnotationPresent(Path.class)) {
      Annotation a = m.getAnnotation(Path.class);
      if (a.url().equals(url)) {
        return m.invoke(c);
      }
    }
  }
  return "404 Page not found";
}
\end{ffcode}
}
\end{pptWide}
\par
\plush{}

\pptSection[DIC]{Dependency Injection Container}
\begin{pptWide}{2}
{\scriptsize\begin{ffcode}
interface Shipment
  int cost();

class Cart
  @Inject private Shipment shmt;
  private Book book;
  void setBook(Book b)
    this.book = b;
  int cost()
    return this.book.price() + this.shmt.cost();

container = new Container();
c = container.make(Cart.class);
c.setBook(new Book("1984"));
x = c.cost();
\end{ffcode}
}
\par\columnbreak\par
{\scriptsize\begin{ffcode}
class Container {
  private HashMap<Class, Object> cache =
    new ConcurrentHashMap<>();
  T make(Class<T> type) {
    // 1. Find @Inject-annotated "shmt" field;
    // 2. Make an instance of "Shipment";
    // 3. Store it in the "cache";
    // 4. Make an instance of "Cart";
    // 5. Store "cart" in the "cache";
    // 6. Assign "shipment" to "cart.shmt";
    // 7. Return "cart".
  }
}
\end{ffcode}
}
How do you think, at the step no.2,
what class will be instantiated?
\end{pptWide}
\par
\plush{}

\pptSection[DI]{Dependency Injection \textit{without} a Container}
\begin{pptWide}{2}
{\scriptsize\begin{ffcode}
interface Shipment
  int cost();

class Cart
  @Inject private Shipment shmt;
  private Book book;
  void setBook(Book b)
    this.book = b;
  int cost()
    return this.book.price() + this.shmt.cost();

container = new Container();
c = container.make(Cart.class);
c.setBook(new Book("1984"));
x = c.cost();
\end{ffcode}
}
\par\columnbreak\par
{\scriptsize\begin{ffcode}
interface Shipment
  int cost();

class Cart
  private final Shipment shmt;
  private final Book book;
  Cart(Shipment s, Book b)
    this.shmt = s;
    this.book = b;
  int cost()
    return this.book.price() + this.shmt.cost();

c = new Cart(new MyShipment(), new Book("1984"));
x = c.cost();
\end{ffcode}
}
\end{pptWide}
\par
\plush{}

\plush{\pptChapter[Types]{Discrimination by Type}}

\pptSection[Polymorphism]{Polymorphism vs. Casting}
\begin{pptWide}{2}
{\small\begin{ffcode}
interface Figure
  void rotate(int d);

class Circle implements Figure
  void rotate(int d) //...
  int radius() //...

class Square implements Figure
  void rotate(int d) //...
  int side() //...
\end{ffcode}
}
\par\columnbreak\par
{\small\begin{ffcode}
// This is polymorphism:
int surface(Figure f)
  return f.surface()

// This is type casting:
int surface(Figure f)
  if (f instanceof Circle c) {
    return c.radius()
  } else if (f instanceof Square s) {
    return s.side() * s.side();
  } else {
    throw new Exception("oops");
  }
\end{ffcode}
}
\end{pptWide}
\par
\plush{}

\plush{\pptChapter[R-and-W]{Read and Watch}}

\href{https://www.yegor256.com/2014/10/03/di-containers-are-evil.html}{Dependency Injection Containers are Code Polluters} by me (2014)

\href{https://www.yegor256.com/2015/04/02/class-casting-is-anti-pattern.html}{Class Casting Is a Discriminating Anti-Pattern} by me (2015)

\href{https://www.yegor256.com/2016/04/12/java-annotations-are-evil.html}{Java Annotations Are a Big Mistake} by me (2016)

\href{https://www.yegor256.com/2022/06/05/reflection-means-hidden-coupling.html}{Reflection Means Hidden Coupling} by me (2022)

\href{https://www.youtube.com/watch?v=cv23Z6xpwDw}{Java Annotations Are a Bad Idea}, at JDK.io conference (2017)

\href{https://www.yegor256.com/2017/11/14/static-factory-methods.html}{Constructors or Static Factory Methods?} by me (2017)

\href{https://www.yegor256.com/2020/11/10/typing-without-types.html}{Strong Typing without Types} by me (2020)

\end{document}

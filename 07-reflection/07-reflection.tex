% SPDX-FileCopyrightText: Copyright (c) 2023-2025 Yegor Bugayenko
% SPDX-License-Identifier: MIT

\documentclass{article}
\usepackage{../lecture-notes/notes}
\newcommand*\thetitle{Reflection}
\newcommand*\thesubtitle{Classpath, Casting, Annotations}
\begin{document}

\lnTitlePage{7}{8}{vYw_GcDkPQA}

\pptToc

\plush{\pptChapter[What?]{What Is Reflection?}}

\begin{lnSnippet}[reflection-example.java]
void print(Object x) {
  if (x instanceof File) {
    System.out.println("Yes!");
  }
}
\end{lnSnippet}
\lnPitch{
  \pptSection[Example]{Simple Example of Reflection in Java}
  \begin{multicols}{2}
  {\small\ffinput{reflection-example.java}}
  \par\columnbreak\par
  ``Reflective computational systems allow computations to observe and
  modify properties of their own behavior. It would be desirable for computations to avail themselves of these
  reflective capabilities, examining themselves in order to make use of
  \ul{meta-level information} in decisions about what to do next.''
  \lnSource{sobel1996introduction}
  \end{multicols}}

\lnQuote
  [Yue Li]
  {yue-li}
  {As reflection is increasingly used in Java programs, the cost of imprecise reflection handling has increased dramatically... Almost all the analyses reported are \ul{unsound} in the presence of reflection since it is either \ul{ignored} or handled \ul{partially}.}
  {li2014self}

\plush{\pptChapter[Casting]{Type Casting and Subsumption}}

\begin{lnSnippet}[casting-wrong.java]
int sizeOf(Iterable items) {
  int size = 0;
  if (items (*@\textcolor{orange}{instanceof}@*) Collection) {
    size = (*@\textcolor{orange}{((Collection)}@*) items).size();
  } else {
    for (Object item : items) {
      ++size;
    }
  }
  return size;
}
\end{lnSnippet}
\begin{lnSnippet}[casting-right.java]
int sizeOf(Iterable items) {
  int size = 0;
  for (Object item : items) {
    ++size;
  }
  return size;
}

int sizeOf((*@\textcolor{orange}{Collection}@*) items) {
  return items.size();
}
\end{lnSnippet}
\lnPitch{
  \pptSection[Iterable]{Iterable \(\to\) Collection}
  \begin{pptWide}{2}
  Downcasting (\textcolor{red}{wrong!}):\par
  {\small\ffinput{casting-wrong.java}}
  \par\columnbreak\par
  Method overloading (\textcolor{green}{right!}):\par
  {\small\ffinput{casting-right.java}}
  \end{pptWide}}

\begin{lnSnippet}[coupling-method.java]
int sizeOf(Iterable items) {
  int size = 0;
  (*@\textcolor{orange}{if}@*) (items instanceof Collection) {
    size = ((Collection) items).size();
  } (*@\textcolor{orange}{else}@*) {
    for (Object item : items) {
      ++size;
    }
  }
  return size;
}
\end{lnSnippet}
\begin{lnSnippet}[coupling-usage.java]
Iterable<Book> books1 =
  new PgBooks1("localhost:5432");
int s1 = (*@\textcolor{orange}{sizeOf}@*)(books1);

Collection<Book> books2 =
  new PgBooks2("localhost:5432");
int s2 = (*@\textcolor{orange}{sizeOf}@*)(books2);
\end{lnSnippet}
\lnPitch{
  \pptSection[Coupling]{Implicit Coupling}
  \begin{pptWide}{2}
  {\small\ffinput{coupling-method.java}}
  \par\columnbreak\par
  {\small\ffinput{coupling-usage.java}}
  \end{pptWide}
  \par
  It may be hard to understand why |s2| is evaluated much faster than |s1|,
  while the signature of |sizeOf()| is the same~\citep{bugayenko2015blog0402}.}

\begin{lnSnippet}[java11-pattern.java]
int sizeOf(Iterable items) {
  int size = 0;
  if (items (*@\textcolor{orange}{instanceof}@*) Collection) {
    size = ((*@\textcolor{orange}{(Collection)}@*) items).size();
  } else {
    for (Object item : items) {
      ++size;
    }
  }
  return size;
}
\end{lnSnippet}
\begin{lnSnippet}[java16-pattern.java]
int sizeOf(Iterable items) {
  int size = 0;
  if ((*@\textcolor{orange}{items instanceof Collection c}@*)) {
    size = c.size();
  } else {
    for (Object item : items) {
      ++size;
    }
  }
  return size;
}
\end{lnSnippet}
\lnPitch{
  \pptSection[JEP-394]{Pattern Matching in Java 16}%
  \begin{pptWide}{2}%
  Java 11 (\textcolor{red}{wrong!}):
  \par
  {\small\ffinput{java11-pattern.java}}
  \par\columnbreak\par
  \pptPin{\raggedleft\includegraphics[width=.3\linewidth]{java-logo.pdf}}%
  Java 16 (\textcolor{red}{\textbf{even worse!}}):
  \par
  {\small\ffinput{java16-pattern.java}}
  \end{pptWide}}

\begin{lnSnippet}[csharp-pattern.cs]
public int sizeOf<T>(IEnumerable<T> items) {
  if (items (*@\textcolor{orange}{is}@*) IList<T> list) {
    return list.Count;
  } else {
    return // count them one by one
  }
}
\end{lnSnippet}
\begin{lnSnippet}[rust-pattern.rs]
enum Color {
  RGB(u8, u8, u8),
  Transparent
}
fn paint(c: Color) {
  (*@\textcolor{orange}{match}@*) c {
    Color::RGB(r, g, b) =>
      println!("#{r}{g}{b}"),
    Color::Transparent =>
      println!("none")
  }
}
fn main() {
  let c = Color::RGB(64, 16, 0);
  paint(c);
}
\end{lnSnippet}
\lnPitch{
  \pptSection[Matching]{C\#, Rust, and Pattern Matching}
  \begin{pptWide}{2}
  C\#:\par
  {\scriptsize\ffinput{csharp-pattern.cs}}
  \par
  Some other languages have \ul{pattern matching} feature,
  including Kotlin, Scala, Haskell, Elixir, Swift, F\#, and Erlang, ...
  which \textcolor{red}{contradicts} the principle of encapsulation.
  \par\columnbreak\par
  Rust:
  \par
  {\scriptsize\ffinput{rust-pattern.rs}}
  \end{pptWide}
  \par}

\plush{\pptChapter[Factory]{Factory Method}}

\begin{lnSnippet}[factory-wrong.java]
interface Figure
  int area();
class Square implements Figure
class Triangle implements Figure
class Polygon implements Figure

class FactoryOfFigures
  Figure make(int sides) {
    if (sides == 3) {
      return (*@\textcolor{orange}{new Triangle()}@*);
    } else if (sides == 4) {
      return (*@\textcolor{orange}{new Square()}@*);
    } else {
      return (*@\textcolor{orange}{new Polygon(sides)}@*);
    }
  }
\end{lnSnippet}
\begin{lnSnippet}[factory-right.java]
class PolymorphicFigure implements Figure
  PolymorphicFigure(int sides)
  @Override int area() {
    if (sides == 3) {
      return (*@\textcolor{orange}{new Triangle().area()}@*);
    } else if (sides == 4) {
      return (*@\textcolor{orange}{new Square().area()}@*);
    } else {
      return (*@\textcolor{orange}{new Polygon(sides).area()}@*);
    }
  }
\end{lnSnippet}
\lnPitch{
  \pptSection[IF]{Conditional Object Construction}
  \begin{pptWide}{2}
  This is \textcolor{red}{wrong}:
  \par
  {\scriptsize\ffinput{factory-wrong.java}}
  \par\columnbreak\par
  This is \textcolor{green}{better}:
  \par
  {\scriptsize\ffinput{factory-right.java}}
  \par
  Here, the semantic of object construction is not visible to
  the client---the \ul{coupling} is ``loose''~\citep{bugayenko2022blog0605}.
  \end{pptWide}}

\begin{lnSnippet}[forname-wrong.java]
interface Figure
  int area();

class Square implements Figure
class Triangle implements Figure
class Polygon implements Figure

class FactoryOfFigures
  Figure make(String name) throws Exception {
    Class<?> c = (*@\textcolor{orange}{Class.forName}@*)(name);
    return c.getConstructor()(*@\textcolor{orange}{.newInstance()}@*);
  }
\end{lnSnippet}
\begin{lnSnippet}[forname-right.java]
class PolymorphicFigure implements Figure
  PolymorphicFigure(String name)
  @Override int area() {
    if ((*@\textcolor{orange}{name.equals}@*)("Triangle")) {
      return new Triangle().area();
    } else if ((*@\textcolor{orange}{name.equals}@*)("Square")) {
      return new Square().area();
    } else {
      return new Polygon(?).area();
    }
  }
\end{lnSnippet}
\lnPitch{
  \pptSection[forName]{Generating class name from a string}
  \begin{pptWide}{2}
  This is \textcolor{red}{wrong}:
  \par
  {\scriptsize\ffinput{forname-wrong.java}}
  \par\columnbreak\par
  This is \textcolor{green}{better}:
  \par
  {\scriptsize\ffinput{forname-right.java}}
  \par
  This is better since the mechanics of class finding
  is explicit---no surprises expected~\citep{bugayenko2017blog1114}.
  \end{pptWide}}

\plush{\pptChapter[Classpath]{Classpath Scanning}}

\begin{lnSnippet}[reflections-interface.java]
interface Foo {}

class Bar implements Foo {}

Reflections rts =
  (*@\textcolor{orange}{new Reflections}@*)("");
Set<Class<?>> types = rts.get(
  SubTypes.of(Foo.class).asClass()
);
\end{lnSnippet}
\begin{lnSnippet}[reflections-annotation.java]
public @interface Foo {}

@Foo
class Bar {}

Reflections rts =
  new Reflections("");
Set<Class<?>> types = rts.get(
  SubTypes.of(
    (*@\textcolor{orange}{TypesAnnotated.with}@*)(Foo.class)
  ).asClass()
);
\end{lnSnippet}
\lnPitch{
  \pptSection[Class]{Finding Java Classes}
  \begin{pptWide}{2}
  {\small\ffinput{reflections-interface.java}}
  \par\columnbreak\par
  {\small\ffinput{reflections-annotation.java}}
  \end{pptWide}
  \par
  The library is called \href{https://github.com/ronmamo/reflections}{Reflections}.
  Instead, use explicit object instantiation.}

\plush{\pptChapter{Annotations}}

\begin{lnSnippet}[annotations-static.java]
interface Pub
  String isbn();

class Book implements Pub
  @Override public String isbn()
    /* ... */
  public (*@\textcolor{orange}{static}@*) String category()
    return "book";

class Journal implements Pub
  @Override public String isbn()
    /* ... */
  public (*@\textcolor{orange}{static}@*) String category()
    return "journal";
\end{lnSnippet}
\begin{lnSnippet}[annotations-annotation.java]
interface Pub
  String isbn();

@Target(ElementType.CLASS)
@Retention(RetentionPolicy.SOURCE)
public @interface Category
  String value();

(*@\textcolor{orange}{@Category("book")}@*)
class Book implements Pub
  @Override public String isbn()
    /* ... */

(*@\textcolor{orange}{@Category("journal")}@*)
class Journal implements Pub
  @Override public String isbn()
    /* ... */
\end{lnSnippet}
\lnPitch{
  \pptSection[Class]{In lieu of static methods}
  \begin{pptWide}{2}
  {\scriptsize\ffinput{annotations-static.java}}
  \par\columnbreak\par
  {\scriptsize\ffinput{annotations-annotation.java}}
  \end{pptWide}
  \par}

\begin{lnSnippet}[path-annotation.java]
@Target(ElementType.METHOD)
@Retention(RetentionPolicy.SOURCE)
public @interface Path
  String url;

class BookController
  (*@\textcolor{orange}{@Path("/book-title")}@*)
  String title()
    // Build HTML page and return it
\end{lnSnippet}
\begin{lnSnippet}[path-dispatch.java]
String dispatch(String url) {
  c = new BooksController();
  b = BooksController.class;
  for (Method m : b(*@\textcolor{orange}{.getDeclaredMethods()}@*)) {
    if (m.isAnnotationPresent(Path.class)) {
      Annotation a = m(*@\textcolor{orange}{.getAnnotation}@*)(Path.class);
      if (a(*@\textcolor{orange}{.url()}@*).equals(url)) {
        return m.invoke(c);
      }
    }
  }
  return "404 Page not found";
}
\end{lnSnippet}
\lnPitch{
  \pptSection[Method]{Locating Methods}
  \begin{pptWide}{2}
  {\small\ffinput{path-annotation.java}}
  \par\columnbreak\par
  {\scriptsize\ffinput{path-dispatch.java}}
  \end{pptWide}
  \par
  The |BookController| doesn't know how exactly the data
  in its annotation is being used and by whom~\citep{bugayenko2016blog0412}.}

\begin{lnSnippet}[dic-usage.java]
interface Shipment
  int cost();

class Cart
  (*@\textcolor{orange}{@Inject}@*) private Shipment shmt;
  private Book book;
  void setBook(Book b)
    this.book = b;
  int cost()
    return this.book.price() + this.shmt.cost();

container = new Container();
c = container(*@\textcolor{orange}{.make(Cart.class)}@*);
c.setBook(new Book("1984"));
x = c.cost();
\end{lnSnippet}
\begin{lnSnippet}[dic-container.java]
class Container {
  private HashMap<Class, Object> cache =
    new ConcurrentHashMap<>();
  T make(Class<T> type) {
    // 1. Find @Inject-annotated "shmt" field;
    // (*@\textcolor{orange}{2. Make an instance of "Shipment";}@*)
    // 3. Store it in the "cache";
    // 4. Make an instance of "Cart";
    // 5. Store "cart" in the "cache";
    // 6. Assign "shipment" to "cart.shmt";
    // 7. Return "cart".
  }
}
\end{lnSnippet}
\lnPitch{
  \pptSection[DIC]{Dependency Injection Container}
  \begin{pptWide}{2}
  {\scriptsize\ffinput{dic-usage.java}}
  \par\columnbreak\par
  {\scriptsize\ffinput{dic-container.java}}
  How do you think, at the step no.2,
  what class will be instantiated?~\citep{bugayenko2014blog1003}
  \end{pptWide}
  \par}

\begin{lnSnippet}[di-container.java]
interface Shipment
  int cost();

class Cart
  (*@\textcolor{orange}{@Inject}@*) private Shipment shmt;
  private Book book;
  void setBook(Book b)
    this.book = b;
  int cost()
    return this.book.price() + this.shmt.cost();

container = new Container();
c = container.make(Cart.class);
c.setBook(new Book("1984"));
x = c.cost();
\end{lnSnippet}
\begin{lnSnippet}[di-constructor.java]
interface Shipment
  int cost();

class Cart
  private (*@\textcolor{orange}{final}@*) Shipment shmt;
  private (*@\textcolor{orange}{final}@*) Book book;
  Cart(Shipment s, Book b)
    this.shmt = s;
    this.book = b;
  int cost()
    return this.book.price() + this.shmt.cost();

c = (*@\textcolor{orange}{new Cart(new MyShipment(), new Book("1984"))}@*);
x = c.cost();
\end{lnSnippet}
\lnPitch{
  \pptSection[DI]{Dependency Injection \textit{without} a Container}
  \begin{pptWide}{2}
  {\scriptsize\ffinput{di-container.java}}
  \par\columnbreak\par
  {\scriptsize\ffinput{di-constructor.java}}
  \end{pptWide}
  \par}

\plush{\pptChapter[Types]{Discrimination by Type}}

\begin{lnSnippet}[polymorphism-classes.java]
interface Figure
  void rotate(int d);
class Circle implements Figure
  void rotate(int d) //...
  int radius() //...
class Square implements Figure
  void rotate(int d) //...
  int side() //...
\end{lnSnippet}
\begin{lnSnippet}[polymorphism-methods.java]
// This is polymorphism:
int area(Figure f)
  return f(*@\textcolor{orange}{.area()}@*);

// This is type casting:
int area(Figure f)
  if (f instanceof Circle c) {
    return c(*@\textcolor{orange}{.radius()}@*)
  } else if (f instanceof Square s) {
    return s(*@\textcolor{orange}{.side()}@*) * s.side();
  } else {
    throw new Exception("oops");
  }
\end{lnSnippet}
\lnPitch{
  \pptSection[Polymorphism]{Polymorphism vs. Casting}
  \begin{pptWide}{2}
  {\small\ffinput{polymorphism-classes.java}}
  \par
  Philosophically speaking, type casting is discrimination~\citep{bugayenko2020blog1110}.
  \par\columnbreak\par
  {\small\ffinput{polymorphism-methods.java}}
  \end{pptWide}
  \par}

\end{document}

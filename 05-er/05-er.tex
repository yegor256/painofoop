% (The MIT License)
%
% Copyright (c) 2023 Yegor Bugayenko
%
% Permission is hereby granted, free of charge, to any person obtaining a copy
% of this software and associated documentation files (the 'Software'), to deal
% in the Software without restriction, including without limitation the rights
% to use, copy, modify, merge, publish, distribute, sublicense, and/or sell
% copies of the Software, and to permit persons to whom the Software is
% furnished to do so, subject to the following conditions:
%
% The above copyright notice and this permission notice shall be included in all
% copies or substantial portions of the Software.
%
% THE SOFTWARE IS PROVIDED 'AS IS', WITHOUT WARRANTY OF ANY KIND, EXPRESS OR
% IMPLIED, INCLUDING BUT NOT LIMITED TO THE WARRANTIES OF MERCHANTABILITY,
% FITNESS FOR A PARTICULAR PURPOSE AND NONINFRINGEMENT. IN NO EVENT SHALL THE
% AUTHORS OR COPYRIGHT HOLDERS BE LIABLE FOR ANY CLAIM, DAMAGES OR OTHER
% LIABILITY, WHETHER IN AN ACTION OF CONTRACT, TORT OR OTHERWISE, ARISING FROM,
% OUT OF OR IN CONNECTION WITH THE SOFTWARE OR THE USE OR OTHER DEALINGS IN THE
% SOFTWARE.

\documentclass{article}
\usepackage{../painofoop}
\newcommand*\thetitle{-ER}
\newcommand*\thesubtitle{...}
\begin{document}

\plush{\poopTitlePage{5}}

\pptToc

\plush{\pptQuote{carlo-pescio.jpg}{When you need a \emph{manager}, it’s often a sign that the \emph{managed} are just plain old data structures and that the manager is the smart procedure doing the real work}{Carlo Pescio \newline \href{http://www.carlopescio.com/2011/04/your-coding-conventions-are-hurting-you.html}{Your Coding Conventions Are Hurting You}, 2011}}

\plush{\pptChapter[Alternatives]{Examples and Alternatives}}

\pptSection{Parser}
\begin{pptWide}{2}
{\small\begin{ffcode}
class Parser {
  static int parseInt(String t) {
    // Parse String into Integer
  }
  static float parseFloat(String t) {
    // Parse String into Float
  }
  // And many more methods...
}

int x = Parser.parseInt("42");
\end{ffcode}
}
\par\columnbreak\par
{\small\begin{ffcode}
class StringAsInt implements Number {
  private final String txt;
  StringAsInt(String t) { this.txt = t; }
  @Override int intValue() {
    // Parse String into Integer
    // and return the value
  }
}

Number n = new StringAsInt("42");
int x = n.intValue();
\end{ffcode}
}
\end{pptWide}
\par
\plush{}

\pptSection{Reader}
\begin{pptWide}{2}
{\small\begin{ffcode}
class Reader {
  static char readChar(InputStream i) {
    // Read the next char from the
    // stream and return it, or NULL
    // if the stream is at the EOF
  }
}

InputStream i = new FileInputStream(..);
char c = Reader.readChar(i);
\end{ffcode}
}
\par\columnbreak\par
{\small\begin{ffcode}
class Chars
  private final InputStream is;
  Chars(InputStream i)
    this.is = i;
  char next()
    // Read the next char from the
    // stream and throw exception
    // if !exists()
  bool exists()
    // Return TRUE if not EOF

InputStream i = new FileInputStream(..);
Chars chars = new Chars(i);
char c = chars.next();
\end{ffcode}
}
\end{pptWide}
\par
\plush{}

\pptSection{Controller}
\begin{pptWide}{2}
{\small\begin{ffcode}
class SimpleController {
  @GET
  @Path("/index")
  HttpResponse index(HttpRequest e) {
    // Build an index page and return
  }
  @POST
  @Path("/update")
  HttpResponse update(HttpRequest e) {
    // Save new user information
    // and return HTTP 303
  }
}
\end{ffcode}
}
\par\columnbreak\par
{\small\begin{ffcode}
class IndexPage implements HttpPage
  HttpResponse process(HttpRequest e) {
    // Build an index page and return
  }
class UpdatePage implements HttpPage
  HttpResponse process(HttpRequest e) {
    // Save new user information
    // and return HTTP 303
  }

new AllPages(
  new IndexPage(),
  new UpdatePage()
);
\end{ffcode}
}
\end{pptWide}
\par
\plush{}

\pptSection{Validator}
\begin{pptWide}{2}
{\small\begin{ffcode}
class Validator {
  bool isValid(int age) {
    return age >= 18;
  }
}
int a = 23;
Validator v = new Validator();
if (!v.isValid(a)) {
  throw new Exception(
    "Age is not valid"
  );
}
\end{ffcode}
}
\par\columnbreak\par
{\scriptsize\begin{ffcode}
interface Age
  int value();
class DefaultAge implements Age
  private final int a;
  DefaultAge(int a)
    this.a = a;
  @Override int value()
    return this.a;
class ValidAge implements Age {
  private final Age origin;
  ValidAge(Age age)
    this.origin = age;
  @Override int value()
    int v = this.origin.value();
    if (v < 18)
      throw new Exception("Age is not valid");
    return v;

Age a = new ValidAge(new DefaultAge(23));
\end{ffcode}
}
\end{pptWide}
\par
\plush{}

\plush{\pptChapter[R\&W]{Read and Watch}}

\href{https://www.yegor256.com/2015/03/09/objects-end-with-er.html}{Don't Create Objects That End With -ER} by me

\end{document}

% SPDX-FileCopyrightText: Copyright (c) 2023-2026 Yegor Bugayenko
% SPDX-License-Identifier: MIT

\documentclass{article}
\usepackage{../lecture-notes/notes}
\usepackage{setspace}
\newcommand*\thetitle{-ER}
\newcommand*\thesubtitle{Alternatives, Clients, MVC}
\begin{document}

\lnTitlePage{5}{8}{6GMiosTLUTc}

\pptToc

\lnQuote
  {carlo-pescio}
  {When you need a \ul{manager}, it’s often a sign that the \ul{managed} are just plain old data structures and that the manager is the smart procedure doing the real work.}
  {pescio2011your}

\plush{\pptChapter[Alternatives]{Examples and Alternatives}}

\begin{lnSnippet}[parser-utils.java]
class Parser {
  static int parseInt(String t) {
    // Parse String into Integer
  }
  static float parseFloat(String t) {
    // Parse String into Float
  }
  // And many more methods...
}

int x = Parser.parseInt("42");
\end{lnSnippet}
\begin{lnSnippet}[parser-oop.java]
class StringAsInt implements Number {
  private final String txt;
  StringAsInt(String t) { this.txt = t; }
  @Override int intValue() {
    // Parse String into Integer
    // and return the value
  }
}

Number n = new StringAsInt("42");
int x = n.intValue();
\end{lnSnippet}
\lnPitch{
  \pptSection{Parser}
  \begin{pptWide}{2}
  {\small\ffinput{parser-utils.java}}
  \par\columnbreak\par
  {\small\ffinput{parser-oop.java}}
  \end{pptWide}}

\begin{lnSnippet}[reader-utils.java]
class Reader {
  static char readChar(InputStream i) {
    // Read the next char from the
    // stream and return it, or NULL
    // if the stream is at the EOF
  }
}

InputStream i = new FileInputStream(..);
char c = Reader.readChar(i);
\end{lnSnippet}
\begin{lnSnippet}[reader-oop.java]
class Chars
  private final InputStream is;
  Chars(InputStream i)
    this.is = i;
  char next()
    // Read the next char from the
    // stream and throw exception
    // if !exists()
  bool exists()
    // Return TRUE if not EOF
InputStream i = new FileInputStream(..);
Chars chars = new Chars(i);
char c = chars.next();
\end{lnSnippet}
\lnPitch{
  \pptSection{Reader}
  \begin{pptWide}{2}
  {\small\ffinput{reader-utils.java}}
  \par\columnbreak\par
  {\small\ffinput{reader-oop.java}}
  \end{pptWide}}

\begin{lnSnippet}[controller-traditional.java]
class SimpleController {
  @GET
  @Path("/index")
  HttpResponse index(HttpRequest e) {
    // Build an index page and return
  }
  @POST
  @Path("/update")
  HttpResponse update(HttpRequest e) {
    // Save new user information
    // and return HTTP 303
  }
}
\end{lnSnippet}
\begin{lnSnippet}[controller-oop.java]
class IndexPage implements HttpPage
  HttpResponse process(HttpRequest e) {
    // Build an index page and return
  }
class UpdatePage implements HttpPage
  HttpResponse process(HttpRequest e) {
    // Save new user information
    // and return HTTP 303
  }
new AllPages(
  new IndexPage(),
  new UpdatePage()
);
\end{lnSnippet}
\lnPitch{
  \pptSection{Controller}
  \begin{pptWide}{2}
  {\small\ffinput{controller-traditional.java}}
  \par\columnbreak\par
  {\small\ffinput{controller-oop.java}}
  \end{pptWide}}

\begin{lnSnippet}[validator-traditional.java]
class Validator {
  bool isValid(int age) {
    return age >= 18;
  }
}
int a = 23;
Validator v = new Validator();
if (!v.isValid(a)) {
  throw new Exception(
    "Age is not valid"
  );
}
\end{lnSnippet}
\begin{lnSnippet}[validator-oop.java]
interface Age
  int value();
class DefaultAge implements Age
  private final int a;
  DefaultAge(int a)
    this.a = a;
  @Override int value()
    return this.a;
class ValidAge implements Age {
  private final Age origin;
  ValidAge(Age age)
    this.origin = age;
  @Override int value()
    int v = this.origin.value();
    if (v < 18)
      throw new Exception("Age is not valid");
    return v;
Age a = new ValidAge(new DefaultAge(23));
\end{lnSnippet}
\lnPitch{
  \pptSection{Validator}
  \begin{pptWide}{2}
  {\small\ffinput{validator-traditional.java}}
  \par\columnbreak\par
  {\scriptsize\ffinput{validator-oop.java}}
  \end{pptWide}}

\begin{lnSnippet}[encoder-utils.java]
package java.net;

class URLEncoder {
  static String encode(String s, String enc) {
    // Encode the string "s" using
    // the "enc" encoding and return
    // the encoded string
  }
}
String e = URLEncoder.encode("@foo");
e.equals("%40foo");
\end{lnSnippet}
\begin{lnSnippet}[encoder-oop.java]
class Encoded implements String {
  private final String origin;
  private final String encoding;
  Encoded(String s, String enc) {
    this.origin = s;
    this.enc = encoding;
  }
  @Override String value() {
    // Encode the string "origin" using
    // the "encoding" and return
    // the encoded string
  }
}
String e = new Encoded("@foo");
e.value().equals("%40foo");
\end{lnSnippet}
\lnPitch{
  \pptSection{Encoder}
  \begin{pptWide}{2}
  {\scriptsize\ffinput{encoder-utils.java}}
  You may want to read more about this in
  my blog~\citep{bugayenko2015blog0309,bugayenko2017blog0912}.
  \par\columnbreak\par
  {\scriptsize\ffinput{encoder-oop.java}}
  \end{pptWide}
  \par
  The right snippet won't work in Java, since \ff{String} is a final class, not an interface, unfortunately.}

\plush{\pptSection[Proof]{Proof: Classes with -ER Suffixes Are More Complex}
  \begin{multicols}{2}
  \includegraphics[width=\linewidth]{er.png}
  \par\columnbreak\par
  \begin{spacing}{0.9}
  ``We took 13,861 Java classes from 212 open source repositories, divided them into three groups (`-Er' classes, `-Utils', and others), and evaluated their complexity. Because average CC and CoCO in the first two groups were almost 3x larger than in the third group, we concluded that functor classes may be considered \ul{bad design}.''
  \end{spacing}
  \par
  \lnSource{sukhova2024java}
  \end{multicols}}

\plush{\pptChapter[-Client]{-Client Suffix}}

\begin{lnSnippet}[aws-client.java]
class AmazonS3Client {
  createBucket(String name);
  deleteBucket(String name);
  doesBucketExist(String name);
  getBucketAcl(String name)
  getBucketPolicy(String name);
  listBuckets();
  // 160+ more methods here
}
client = new AmazonS3Client("us-1");
client.createBucket("foo");
client.putObject("foo", "a.txt");
client.writeObject("foo", "a.txt", "data");
\end{lnSnippet}
\begin{lnSnippet}[aws-oop.java]
region = new S3Region("us-1");
bucket = region.createBucket("foo");
object = bucket.putObject("a.txt");
object.write("data");
\end{lnSnippet}
\lnPitch{
  \pptSection[AWS]{AWS Java Client}
  \begin{pptWide}{2}
  {\scriptsize\ffinput{aws-client.java}}
  \par\columnbreak\par
  {\scriptsize\ffinput{aws-oop.java}}
  It's here: \href{https://github.com/jcabi/jcabi-s3}{jcabi/jcabi-s3}.
  \end{pptWide}
  \par
  The left snippet is:
  \begin{inparaenum}[1)]
  \item procedural,
  \item hard to test,
  \item resembles a utility class,
  and
  \item is hard to extend.
  \end{inparaenum}
  The right one is object-oriented.}

\plush{\pptChapter[Performance]{What About Performance?}}

\begin{lnSnippet}[sticky-original.java]
class StringAsInt implements Number {
  private final String txt;
  StringAsInt(String t) { this.txt = t; }
  @Override int intValue() {
    // Parse String into Integer
    // and return the value
  }
}

Number n = new StringAsInt("42");
int x = n.intValue();
\end{lnSnippet}
\begin{lnSnippet}[sticky-cached.java]
class StickyInt implements Number {
  private final Number origin;
  private int cache = 0;
  private bool cached = false;
  StickyInt(Number n) { origin = n; }
  @Override int intValue() {
    if (!cached) {
      cache = origin.intValue();
      cached = true;
    }
    return cache; } }
\end{lnSnippet}
\lnPitch{
  \pptSection[Sticky]{Sticky Parseable Object}
  \begin{pptWide}{2}
  {\small\ffinput{sticky-original.java}}
  \par\columnbreak\par
  {\small\ffinput{sticky-cached.java}}
  \end{pptWide}
  \par
  Is it thread-safe though?}

\begin{lnSnippet}[threadsafe-unsafe.java]
class StickyInt implements Number {
  private final Number origin;
  private int cache = 0;
  private bool cached = false;
  StickyInt(Number n) { origin = n; }
  @Override int intValue() {
    if (!cached) {
      cache = origin.intValue();
    }
    return cache;
  }
}
\end{lnSnippet}
\begin{lnSnippet}[threadsafe-safe.java]
class StickyInt implements Number {
  private final Number origin;
  private final AtomicReference<Integer> cache =
    new AtomicReference<Integer>(null);
  StickyInt(Number n) { origin = n; }
  @Override int intValue() {
    return cache.updateAndGet(
      x -> {
        if (x == null) {
          return origin.intValue();
        }
        return x;
      }
    );
  }
}
\end{lnSnippet}
\lnPitch{
  \pptSection[Safe]{Thread-safe Sticky Parseable Object}
  \begin{pptWide}{2}
  {\small\ffinput{threadsafe-unsafe.java}}
  \par\columnbreak\par
  {\scriptsize\ffinput{threadsafe-safe.java}}
  \end{pptWide}
  \par
  The left snippet is not thread-safety, while the right one is.}

\plush{\pptChapter[MVC]{Model-View-Controller (MVC)}}

\begin{lnSnippet}[mvc-controller.java]
class Controller {
  @GET
  @Path("/b{id}")
  String index(int id) {
    Book book = em.findById(id);
    View v = new HtmlView("book.html");
    v.set("title", book.getTitle());
    return v.renderHtml();
  }
}
\end{lnSnippet}
\lnPitch{
  \pptSection[Controller]{The Controller}
  \begin{pptWide}{2}
  {\small\ffinput{mvc-controller.java}}
  \par\columnbreak\par
  \pptPic{0.7}{mvc.png}\par
  This is bad OOP~\citep{bugayenko2016blog1213}.
  \end{pptWide}}

\begin{lnSnippet}[html-controller.java]
class Controller {
  @GET
  @Path("/b{id}")
  String index(int id) {
    Book book = em.findById(id);
    View v = new HtmlView("book.html");
    v.set("title", book.getTitle());
    return v.renderHtml();
  }
}
\end{lnSnippet}
\begin{lnSnippet}[html-oop.java]
interface Book
  String title();
class PgBook implements Book
  String title() // loads from PostgreSQL
interface Page
  String html();
class HtmlBook implements Book, Page
  String html() // renders in HTML
  String title() // returns origin.title()
class PageOnPath implements Page
  private final String path;
  private final Page origin;
  String html() // renders if path matches
\end{lnSnippet}
\lnPitch{
  \pptSection[HTML]{Book as HTML}
  \begin{pptWide}{2}
  {\small\ffinput{html-controller.java}}
  \par\columnbreak\par
  {\scriptsize\ffinput{html-oop.java}}
  \end{pptWide}
  \par
  Check \href{https://github.com/yegor256/jpages}{yegor256/jpages} and \href{https://github.com/yegor256/takes}{yegor256/takes}.}

\plush{\begin{pptMiddle}
  \pptChapter[Takes]{Rultor + Takes}
  \par
  \begin{tabular}{lp{2em}l}
    \includegraphics[height=1.6in]{rultor.png} & & \includegraphics[height=1.6in]{takes.png} \\
    \href{https://www.rultor.com}{\texttt{rultor.com}} & & \href{https://www.takes.org}{\texttt{takes.org}} \\
  \end{tabular}
\end{pptMiddle}}

\plush{\begin{multicols}{2}
  \includegraphics[width=.9\linewidth]{rultor-1.png}
  \par\columnbreak\par
  \includegraphics[width=.9\linewidth]{rultor-2.png}
  \end{multicols}\par
  {\tiny \url{https://github.com/yegor256/rultor/blob/master/src/main/java/com/rultor/web/TkApp.java}\par}}

\end{document}

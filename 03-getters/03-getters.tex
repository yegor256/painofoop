% (The MIT License)
%
% Copyright (c) 2023 Yegor Bugayenko
%
% Permission is hereby granted, free of charge, to any person obtaining a copy
% of this software and associated documentation files (the 'Software'), to deal
% in the Software without restriction, including without limitation the rights
% to use, copy, modify, merge, publish, distribute, sublicense, and/or sell
% copies of the Software, and to permit persons to whom the Software is
% furnished to do so, subject to the following conditions:
%
% The above copyright notice and this permission notice shall be included in all
% copies or substantial portions of the Software.
%
% THE SOFTWARE IS PROVIDED 'AS IS', WITHOUT WARRANTY OF ANY KIND, EXPRESS OR
% IMPLIED, INCLUDING BUT NOT LIMITED TO THE WARRANTIES OF MERCHANTABILITY,
% FITNESS FOR A PARTICULAR PURPOSE AND NONINFRINGEMENT. IN NO EVENT SHALL THE
% AUTHORS OR COPYRIGHT HOLDERS BE LIABLE FOR ANY CLAIM, DAMAGES OR OTHER
% LIABILITY, WHETHER IN AN ACTION OF CONTRACT, TORT OR OTHERWISE, ARISING FROM,
% OUT OF OR IN CONNECTION WITH THE SOFTWARE OR THE USE OR OTHER DEALINGS IN THE
% SOFTWARE.

\documentclass{article}
\usepackage{../painofoop}
\newcommand*\thetitle{Getters}
\newcommand*\thesubtitle{Accessors, Encapsulation, Printers}
\begin{document}

\plush{\poopTitlePage{3}}

\pptToc

\plush{\pptChapter{What is a Getter?}}

\pptSection[What?]{Which one is a getter?}
\begin{pptWide}{2}
{\small\begin{ffcode}
class MiddleSquare {
  private short seed;
  MiddleSquare(short s) { seed = s; }
  short getNextNumber() {
    int s = (int) seed * seed;
    seed = (short) (s >> 8);
    return seed;
  }
}
\end{ffcode}
}
\par\columnbreak\par
{\small\begin{ffcode}
class MiddleSquare {
  private short seed;
  MiddleSquare(short s) { seed = s; }
  short getSeed() {
    return this.seed;
  }
}
\end{ffcode}
}
\end{pptWide}
\par
A getter method may also be known as an ``\emph{accessor}.''
\plush{}

\pptSection[Boilerplate]{Boilerplate getters}
\begin{pptWide}{2}
{\small\begin{ffcode}
class Document
  def initialize(n)
    @name = n
  end
  def name
    @name
  end
end
d = Document.new("/tmp/test.txt")
puts d.name
\end{ffcode}
}
\par\columnbreak\par
{\small\begin{ffcode}
class Document
  attr_reader :name
  def initialize(n)
    @name = n
  end
end
d = Document.new("/tmp/test.txt")
puts d.name
\end{ffcode}
}
\end{pptWide}
\par
Modern programming languages often offer the ability to generate the \emph{boilerplate} for mutators and accessors in a single line.
\plush{}

\pptSection[Purpose]{What getters are for?}
\begin{pptWide}{2}
\pptPic{0.8}{gpt-1.png}
\par\columnbreak\par
\pptPic{0.8}{gpt-2.png}
\end{pptWide}
\par
This is what GPT-4 replied to my question: ``Why should we use getters in Java instead of making class fields public? What are the benefits?''
\plush{}

\pptSection[DTO]{Data Transfer Objects (DTO)}
\begin{pptWide}{2}
{\small\begin{ffcode}
class BookDTO {
  private int id;
  private String author;
  private String title;
  BookDTO(int i, String a, String t)
    { id = i; author = a; title = t; }
  int getId() { return id; }
  String getAuthor() { return author; }
  String getTitle() { return title; }
}
\end{ffcode}
}
\par\columnbreak\par
{\small\begin{ffcode}
class JsonApi {
  BookDTO getById(int id) { /* ... */ }
}
BookDTO dto = api.getById(42);
print(dto.getTitle());
print(dto.getAuthor());
\end{ffcode}
}
\end{pptWide}
\par
\plush{}

\plush{\pptChapter{Encapsulation}}

\plush{\begin{pptMiddle}\pptQuote{grady-booch.jpg}{Encapsulation is most often achieved through \emph{information hiding}, which
is the process of hiding all the \emph{secrets} of an object that
do not contribute to its essential characteristics; typically,
the structure of an object is hidden, as well as the implementation of its methods.

\vspace{12pt}
Encapsulation provides \emph{explicit barriers} among different abstractions and thus leads to a clear separation of concerns.
}{Grady Booch et al. (2007) \newline Object-Oriented Analysis and Design with Applications}\end{pptMiddle}}

\plush{\begin{pptMiddle}\pptQuote{../01-algorithms/david-west.jpg}{In most ways, encapsulation is a \emph{discipline} more than a real barrier; seldom is the \emph{integrity} of an object protected in any absolute sense, and this is especially true of software objects, so it is up to the user of an object to \emph{respect} that object’s encapsulation}{David West et al. (2006) \newline Object Thinking}\end{pptMiddle}}

\pptSection[Integrity]{Who is protecting the integrity better?}
\begin{pptWide}{2}
{\small\begin{ffcode}
class User {
  private String name;
  User(String n) { name = n; }
  int getName() {
    return this.name;
  }
}
if (employees.contains(user.getName())) {
  /* pay a salary */
}
\end{ffcode}
}
\par\columnbreak\par
{\small\begin{ffcode}
class User {
  private String name;
  User(String n) { name = n; }
  bool isEmployee() {
    return employees.contains(this.name);
  }
}
if (user.isEmployee()) {
  /* pay a salary */
}
\end{ffcode}
}
\end{pptWide}
\par
Which class hides data better?
\plush{}

\pptSection[Semantic]{Who knows about data semantic?}
\begin{pptWide}{2}
{\small\begin{ffcode}
class Box {
  private int weight;
  Box(int kg) { weight = kg; }
  int getWeight() {
    return this.weight;
  }
}
int w = box.getWeight();
int lbs = w / 0.454;
printf("The weight is \%d lbs\n");
\end{ffcode}
}
\par\columnbreak\par
{\small\begin{ffcode}
class Box {
  private int weight;
  Box(int kg) { weight = kg; }
  int getLbs() {
    return this.weight / 0.454;
  }
}
int lbs = box.getLbs();
printf("The weight is \%d lbs\n");
\end{ffcode}
}
\end{pptWide}
\par
What happens if the \ff{Box} decides to store the \ff{weight} in pounds instead of kilograms? How will it know how many of its clients still assume that the \ff{weight} is in kilograms?
\plush{}

\plush{\pptChapter{The Alternative}}

\pptSection[TellDontAsk]{Tell Don't Ask}
\begin{pptWide}{2}
{\small\begin{ffcode}
class BookDTO {
  private int id;
  private String author;
  private String title;
  BookDTO(int i, String a, String t)
    { id = i; author = a; title = t; }
  void print() { /* ... */ }
}
\end{ffcode}
}
\par\columnbreak\par
{\small\begin{ffcode}
class JsonApi {
  BookDTO getById(int id) { /* ... */ }
}
BookDTO dto = api.getById(42);
dto.print();
\end{ffcode}
}
\end{pptWide}
\par
\plush{}

\pptSection[NoPrefix]{Get rid of the ``get'' prefix}
\begin{pptWide}{2}
{\small\begin{ffcode}
class Book {
  private int id;
  private String author;
  private String title;
  BookDTO(int i, String a, String t)
    { id = i; author = a; title = t; }
  int id() { return id; }
  String author() { return author; }
  String title() { return title; }
}
\end{ffcode}
}
\par\columnbreak\par
{\small\begin{ffcode}
class JsonApi {
  Book getById(int id) { /* ... */ }
}
Book book = api.getById(42);
print(book.author());
print(book.title());
\end{ffcode}
}
\end{pptWide}
\par
\plush{}

\pptSection[NoPrefix]{Make fields public}
\begin{pptWide}{2}
{\small\begin{ffcode}
class Book {
  private int id;
  private String author;
  private String title;
  BookDTO(int i, String a, String t)
    { id = i; author = a; title = t; }
}
\end{ffcode}
}
\par\columnbreak\par
{\small\begin{ffcode}
class JsonApi {
  Book getById(int id) { /* ... */ }
}
Book book = api.getById(42);
print(book.author);
print(book.title);
\end{ffcode}
}
\end{pptWide}
\par
\plush{}

\plush{\pptChapter{Read and Watch}}

\href{https://martinfowler.com/bliki/GetterEradicator.html}{GetterEradicator} by Martin Fowler

\href{TellDontAsk}{https://martinfowler.com/bliki/TellDontAsk.html} by Martin Fowler

\href{https://www.infoworld.com/article/2073723/why-getter-and-setter-methods-are-evil.html}{Why getter and setter methods are evil} by Allen Holub

\href{https://www.yegor256.com/2014/09/16/getters-and-setters-are-evil.html}{Getters/Setters. Evil. Period.} by me

\end{document}

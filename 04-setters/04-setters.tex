% (The MIT License)
%
% Copyright (c) 2023-2024 Yegor Bugayenko
%
% Permission is hereby granted, free of charge, to any person obtaining a copy
% of this software and associated documentation files (the 'Software'), to deal
% in the Software without restriction, including without limitation the rights
% to use, copy, modify, merge, publish, distribute, sublicense, and/or sell
% copies of the Software, and to permit persons to whom the Software is
% furnished to do so, subject to the following conditions:
%
% The above copyright notice and this permission notice shall be included in all
% copies or substantial portions of the Software.
%
% THE SOFTWARE IS PROVIDED 'AS IS', WITHOUT WARRANTY OF ANY KIND, EXPRESS OR
% IMPLIED, INCLUDING BUT NOT LIMITED TO THE WARRANTIES OF MERCHANTABILITY,
% FITNESS FOR A PARTICULAR PURPOSE AND NONINFRINGEMENT. IN NO EVENT SHALL THE
% AUTHORS OR COPYRIGHT HOLDERS BE LIABLE FOR ANY CLAIM, DAMAGES OR OTHER
% LIABILITY, WHETHER IN AN ACTION OF CONTRACT, TORT OR OTHERWISE, ARISING FROM,
% OUT OF OR IN CONNECTION WITH THE SOFTWARE OR THE USE OR OTHER DEALINGS IN THE
% SOFTWARE.

\documentclass{article}
\usepackage{../painofoop}
\newcommand*\thetitle{Setters}
\newcommand*\thesubtitle{Mutability, Problems, DTO and ORM}
\begin{document}

\plush{\poopTitlePage{4}{h0jf9E1tjf4}}

\pptToc

\plush{\pptChapter{Pre-Test}}

\pptSection[How?]{How would you do this?}
\begin{pptWide}{2}
{\small\begin{ffcode}
Message m = new Message();
m.(*@\textcolor{orange}{setName}@*)("Sarah");
m.print(); // Hello, Sarah!

m.(*@\textcolor{orange}{setName}@*)("Victor");
m.print(); // Hello, Victor!

m.(*@\textcolor{orange}{setName}@*)("Leyla");
m.print(); // Hello, Leyla!
\end{ffcode}
}
\par\columnbreak\par
{\small\begin{ffcode}
Message m1 = (*@\textcolor{orange}{new}@*) Message("Sarah");
m1.print(); // Hello, Sarah!

Message m2 = (*@\textcolor{orange}{new}@*) Message("Victor");
m2.print(); // Hello, Victor!

Message m3 = (*@\textcolor{orange}{new}@*) Message("Leyla");
m3.print(); // Hello, Leyla!
\end{ffcode}
}
\end{pptWide}
\par
\plush{}

\plush{\pptChapter{Mutability}}

\pptSection[Definition]{Which object is \ul{im}mutable?}
\begin{pptWide}{2}
{\small\begin{ffcode}
class Book {
  private String title;
  Book(String t) { title = t; }
  void setTitle(String t) {
    this.title = t;
  }
  String getTitle() {
    return this.title;
  }
}
b = new Book();
b.(*@\textcolor{orange}{setTitle}@*)("Object Thinking");
\end{ffcode}
}
\par\columnbreak\par
{\small\begin{ffcode}
class Book {
  private final String title;
  Book(String t) { title = t; }
  void withTitle(String t) {
    return new Book(t);
  }
  String getTitle() {
    return this.title;
  }
}
b1 = new Book();
b2 = b1.(*@\textcolor{orange}{withTitle}@*)("Object Thinking");
\end{ffcode}
}
\end{pptWide}
\par
\plush{}

\print{\pptSection[Gradients]{There are four gradients of immutability}}
\print{I. Constant}
\print{II. Not a Constant}
\print{III. Represented Mutability}
\print{IV. Encapsulated Mutability}
\plush{{\small You may read my blog about this~\citep{bugayenko2016blog0907}.\par}}

\pptSection[Constant]{Gradient I: Constant}
\begin{pptWide}{2}
{\small\begin{ffcode}
class Book {
  private final String t;
  Book(String t) { this.t = t; }
  String title() {
    (*@\textcolor{orange}{return this.t}@*);
  }
}
\end{ffcode}
}
\par\columnbreak\par
{\small\begin{ffcode}
Book b = new Book("Object Thinking");
String t1 = b.title();
String t2 = b.title()
\end{ffcode}
}
\par
The \ff{title()} method returns exactly the same data on each call.
This object is definitely \ul{immutable}.
\end{pptWide}
\par
\plush{}

\pptSection[NotConstant]{Gradient II: Not a Constant}
\begin{pptWide}{2}
{\small\begin{ffcode}
class Book {
  private final String t;
  Book(String t) { this.t = t; }
  String title() {
    return String.format(
      "%s / %s", title, (*@\textcolor{orange}{return new Date()}@*)
    );
  }
}
\end{ffcode}
}
\par\columnbreak\par
{\small\begin{ffcode}
Book b = new Book("Object Thinking");
String t1 = b.title();
String t2 = b.title()
\end{ffcode}
}
\par
The \ff{title()} method returns different data on each new call,
depending on \ul{system timer}.
Does it make the object mutable or not?
\end{pptWide}
\par
\plush{}

\pptSection[Represented]{Gradient III: Represented Mutability}
\begin{pptWide}{2}
{\scriptsize\begin{ffcode}
class Book {
  private final Path path;
  Book(Path p) { this.path = p; }
  Book rename(String title) {
    Files.write(
      this.path,
      title.getBytes(),
      StandardOpenOption.CREATE
    );
    return this;
  }
  String title() {
    return new String(
      Files.(*@\textcolor{orange}{readAllBytes}@*)(this.path)
    );
  }
}
\end{ffcode}
}
\par\columnbreak\par
{\small\begin{ffcode}
Book b = new Book("Object Thinking");
String t1 = b.title();
b.rename("Elegant Objects");
String t2 = b.title()
\end{ffcode}
}
\par
The \ff{title()} method returns different data on each new call,
depending on the content of the \ul{file} in the \ul{file system}.
Does it make the object mutable or not?
\end{pptWide}
\par
\plush{}

\pptSection[Encapsulated]{Gradient IV: Encapsulated Mutability}
\begin{pptWide}{2}
{\small\begin{ffcode}
class Book {
  private final (*@\textcolor{orange}{StringBuffer}@*) buffer;
  Book rename(String t) {
    this.buffer.setLength(0);
    this.buffer.append(t);
    return this;
  }
  String title() {
    return this.buffer.toString();
  }
}
\end{ffcode}
}
\par\columnbreak\par
{\small\begin{ffcode}
Book b = new Book("Object Thinking");
String t1 = b.title();
b.rename("Elegant Objects");
String t2 = b.title()
\end{ffcode}
}
\par
The \ff{title()} method returns different data on each new call,
depending on the content of the \ul{memory block}.
Does it make the object mutable or not?
\end{pptWide}
\par
\plush{}

\plush{
\pptThought{Only gradients III and IV cause problems, while ``Constant'' and ``Not a Constant'' objects are harmless.}
\par
{\small You may want to read my blog about immutability~\citep{bugayenko2014blog0609,bugayenko2014blog1222,bugayenko2014blog1209,bugayenko2014blog1107}.\par}}

\plush{\pptChapter[Problems]{Drawbacks of Mutability}}

\qte
  {joshua-bloch}
  {Immutable classes are \ul{easier to design}, implement, and use than mutable classes. They are \ul{less prone to error} and are \ul{more secure}.}
  {bloch2008effective}

\qte
  [Steve Freeman]
  {steve-freeman}
  {Writing large-scale functional programs is a topic for a different book, but we find that a little immutability within the implementation of a class leads to \ul{much safer} code and that, if we do a good job, the code \ul{reads well} too.}
  {freeman2009growing}

\pptSection[Side-effects]{1) Side effects}
\begin{pptWide}{2}
With a side effect:\par
{\small\begin{ffcode}
public String post(Request request) {
  request.(*@\textcolor{orange}{setMethod}@*)("POST");
  return request.fetch();
}

r = new Request("x.com");
r.setMethod("GET");
String first = this.post(r);

String second = r.fetch();
\end{ffcode}
}
\par\columnbreak\par
Without a side effect:\par
{\small\begin{ffcode}
public String post(Request request) {
  return request
    .(*@\textcolor{orange}{withMtd}@*)("POST")
    .fetch();
}

r = new Request("x.com").withMtd("GET");
String first = this.post(r);

String second = r.fetch();
\end{ffcode}
}
\end{pptWide}
\par
\plush{}

\pptSection[Concurrency]{2) Thread (un-)safety}
\begin{pptWide}{2}
{\small\begin{ffcode}
class Books {
  private (*@\textcolor{orange}{int}@*) c = 0;
  void add() {
    this.c = this.c + 1;
  }
}
\end{ffcode}
\begin{tabular}{l>{\raggedright}p{11cm}}%
  \raisebox{-0.9\height}{\pptPic{0.2}{goetz.png}} & \small \citet{goetz2006java} explained the advantages of immutable objects
  in more details in their very famous book ``Java Concurrency in Practice'' (highly recommended!) \\
\end{tabular}
}
\par\columnbreak\par
{\small\begin{ffcode}
ExecutorService e =
  Executors.newCachedThreadPool();
final Books books = new Books();
for (int i = 0; i < 1000; i++) {
  e.execute(
    new Thread(
      () -> {
        books.add();
      }
    )
  );
}
// What is the value of "books.c"?
\end{ffcode}
}
\end{pptWide}
\par
\plush{}

\pptSection[Coupling]{3) Temporal Coupling}
\begin{pptWide}{2}
{\small\begin{ffcode}
r = new Request("x.com");
r.setMethod("POST");
String first = r.fetch();
r.setBody("text=hello");
String second = r.fetch();
\end{ffcode}
}
\par\columnbreak\par
{\small\begin{ffcode}
r = new Request("x.com");

// 100 lines later:
// r.setMethod("POST");
// String first = r.fetch();

r.setBody("text=hello");
String second = r.fetch();
\end{ffcode}
}
\end{pptWide}
\par
``\ul{Sequential} coupling (also known as \ul{temporal} coupling) is a form of coupling where a class requires its methods to be called in a particular sequence.'' --- \href{https://en.wikipedia.org/wiki/Sequential_coupling}{Wikipedia}.
\plush{}

\qte
  {robert-martin}
  {Side effects are \ul{lies}. Your function promises to do one thing, but it also does other hidden things. They are devious and damaging mistruths that often result in strange \ul{temporal couplings} and order dependencies.}
  {martin2008clean}

\qte
  {steve-mcconnell}
  {Sequential cohesion (considered to be less than ideal) exists when a routine contains operations that must be performed in a \ul{specific order}, that share data from step to step, and that don't make up a complete function when done together.}
  {mcconnell1998}

\qte
  {michael-feathers}
  {Back in the early days of programming, this was named temporal coupling, and it is a pretty \ul{nasty thing} when you do it excessively. When you group things together just because they have to happen at the same time, the relationship between them isn’t very strong. Later you might find that you have to do one of those things without the other, but at that point they might have grown together. Without a seam, separating them can be hard work.}
  {feathers2004working}

\pptSection[Identity]{4) Identity Mutability}
\begin{pptWide}{2}
{\small\begin{ffcode}
Map<Date, String> map = new HashMap<>();
Date date = new Date();
map.put(date, "hello, world!");

// This is (*@\textcolor{green}{TRUE}@*):
assert map.containsKey(date);

date.setTime(12345L);
// (*@\textcolor{red}{Why this is FALSE??}@*):
assert map.containsKey(date);
\end{ffcode}
}
\par\columnbreak\par
``In order for an object to be shared safely, it must be immutable: it cannot be changed except by full replacement.''\par
\source{evans2004domain}
\end{pptWide}
\plush{}

\qte
  {eric-evans}
  {As long as a \ul{value object} is immutable, change management is simple---there isn't any change except full replacement. Immutable objects can be freely shared.}
  {evans2004domain}

\plush{\pptChapter{ORM}}

\plush{
  \pptThought{ORM stands for ``Object Relational Mapping,'' which is an attempt to represent a relational data model in objects and relations between them, such as attributes, methods, and inheritance}
  \par
  {\small You may want to read my blog about ORM~\citep{bugayenko2014blog1201}.\par}}

\qte
  {ted-neward}
  {One of the first and most easily-recognizable problems in using objects as a front-end to a relational data store is that of \ul{how to map} classes to tables.}
  {neward2006vietnam}

\pptSection[JPA]{Java Persistence API}
\begin{pptWide}{2}
{\small\begin{ffcode}
@Entity
@Table(name = "movie")
public class Movie {
  @Id
  private Long id;
  private String name;
  private Integer year;
  // ctors
  // getters
  // setters
}
\end{ffcode}
}
\par\columnbreak\par
{\small\begin{ffcode}
EntityManager em = getEntityManager();
em.getTransaction().begin();
Movie movie = em.(*@\textcolor{orange}{findById}@*)(1L);
movie.setName("The Godfather");
em.(*@\textcolor{orange}{persist}@*)(movie);
em.getTransaction().commit();
\end{ffcode}
}
\end{pptWide}
\par
\plush{}

\pptSection[SQL-speaking]{SQL speaking objects}
\begin{pptWide}{2}
{\small\begin{ffcode}
interface Movie {
  int id();
  String title();
  String author();
}
Movie m = new PgMovie(ds, 1L);
m.(*@\textcolor{orange}{rename}@*)("The Godfather");
\end{ffcode}
}
\par
Here I'm using \href{https://github.com/jcabi/jcabi-jdbc}{jcabi-jdbc},
an object-oriented wrapper around JDBC data source.
\par\columnbreak\par
{\scriptsize\begin{ffcode}
final class PgMovie implements Movie
  private final Source dbase;
  private final int number;
  public PgMovie(DataSource data, int id)
    this.dbase = data;
    this.number = id;
  public String title()
    return new JdbcSession(this.dbase)
      .sql("SELECT title FROM movie WHERE id = ?")
      .set(this.number)
      .select(new SingleOutcome<String>(String.class));
  public void (*@\textcolor{orange}{rename}@*)(String n)
    new JdbcSession(this.dbase)
      .sql("UPDATE movie SET name = ? WHERE id = ?")
      .set(n)
      .set(this.number)
      .execute();
\end{ffcode}
}
\end{pptWide}
\par
\plush{}

\pptSection[JOINs]{Complex SQL queries}
\begin{pptWide}{2}
{\small\begin{ffcode}
final class PgMovies
  private final Source dbase;
  public PgMovies(DataSource data)
    this.dbase = data;
  public Movie movie(Long id)
    return new PgMovie(this.dbase, id);
\end{ffcode}
}
\par\columnbreak\par
{\scriptsize\begin{ffcode}
final class PgMovie implements Movie
  private final Source dbase;
  private final int number;
  public PgMovie(DataSource data, int id)
    this.dbase = data;
    this.number = id;
  public String title()
    return new JdbcSession(this.dbase)
      .sql("SELECT title FROM movie WHERE id = ?")
      .set(this.number)
      .select(new SingleOutcome<String>(String.class));
  public String author()
    return new JdbcSession(this.dbase)
      .sql("(*@\textcolor{orange}{SELECT name FROM movie JOIN author ON author.id = movie.author WHERE movie.id = ?}@*)")
      .set(this.number)
      .select(new SingleOutcome<String>(String.class));
\end{ffcode}
}
\end{pptWide}
\plush{}

\plush{\pptChapter[Apache]{Apache Commons Email}}

\pptSection[Email]{\texttt{org.apache.commons.mail.Email}}
\begin{pptWide}{2}
{\scriptsize\begin{ffcode}
public abstract class Email {
  protected MimeMessage message;
  protected String charset;
  protected InternetAddress fromAddress;
  protected String subject;
  protected MimeMultipart emailBody;
  protected Object content;
  protected String contentType;
  protected boolean debug;
  protected Date sentDate;
  protected Authenticator authenticator;
  protected String hostName;
  protected String smtpPort;
  protected String sslSmtpPort;
  protected List<InternetAddress> toList;
  protected List<InternetAddress> ccList;
  protected List<InternetAddress> bccList;
  protected List<InternetAddress> replyList;
  protected String bounceAddress;
  protected Map<String, String> headers;
  protected boolean popBeforeSmtp;
  protected String popHost;
  protected String popUsername;
  protected String popPassword;
  protected boolean tls;
  protected boolean ssl;
  protected int socketTimeout;
  protected int socketConnectionTimeout;
  private boolean startTlsEnabled;
  private boolean startTlsRequired;
  private boolean sslOnConnect;
  private boolean sslCheckServerIdentity;
  private boolean sendPartial;
  private Session session;
}
\end{ffcode}
}
\end{pptWide}
\par
{\tiny\url{https://github.com/apache/commons-email/blob/EMAIL_1_5/src/main/java/org/apache/commons/mail/Email.java}\par}
\plush{}

\plush{\pptChapter[Performance]{What About Performance?}}

\qte
  [\nospell{Zoran Budimli{\'c}}]
  {zoran-budimlic}
  {Although Java implementations have been made great strides, they still \ul{fall short} on programs that use the full power of Java's object-oriented features. Ideally, future compiler technologies will be able to \ul{automatically transform} the [OO style code] into something that approaches the [procedural style] in performance.}
  {budimlic1999cost}

\end{document}

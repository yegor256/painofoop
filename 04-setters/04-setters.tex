% (The MIT License)
%
% Copyright (c) 2023 Yegor Bugayenko
%
% Permission is hereby granted, free of charge, to any person obtaining a copy
% of this software and associated documentation files (the 'Software'), to deal
% in the Software without restriction, including without limitation the rights
% to use, copy, modify, merge, publish, distribute, sublicense, and/or sell
% copies of the Software, and to permit persons to whom the Software is
% furnished to do so, subject to the following conditions:
%
% The above copyright notice and this permission notice shall be included in all
% copies or substantial portions of the Software.
%
% THE SOFTWARE IS PROVIDED 'AS IS', WITHOUT WARRANTY OF ANY KIND, EXPRESS OR
% IMPLIED, INCLUDING BUT NOT LIMITED TO THE WARRANTIES OF MERCHANTABILITY,
% FITNESS FOR A PARTICULAR PURPOSE AND NONINFRINGEMENT. IN NO EVENT SHALL THE
% AUTHORS OR COPYRIGHT HOLDERS BE LIABLE FOR ANY CLAIM, DAMAGES OR OTHER
% LIABILITY, WHETHER IN AN ACTION OF CONTRACT, TORT OR OTHERWISE, ARISING FROM,
% OUT OF OR IN CONNECTION WITH THE SOFTWARE OR THE USE OR OTHER DEALINGS IN THE
% SOFTWARE.

\documentclass{article}
\usepackage{../painofoop}
\newcommand*\thetitle{Setters}
\newcommand*\thesubtitle{...}
\begin{document}

\plush{\poopTitlePage{1}}

\pptToc

\plush{\pptChapter{Mutability}}

\pptSection[Definition]{Which object is immutable?}
\begin{pptWide}{2}
{\small\begin{ffcode}
class Book {
  private String title;
  Book(String t) { title = t; }
  void setTitle(String t) {
    this.title = t;
  }
  String getTitle() {
    return this.title;
  }
}
Book b = new Book("Object Thinking");
b.setTitle("It");
\end{ffcode}
}
\par\columnbreak\par
{\small\begin{ffcode}
class Book {
  private final String title;
  Book(String t) { title = t; }
  void withTitle(String t) {
    return new Book(t);
  }
  String getTitle() {
    return this.title;
  }
}
Book b1 = new Book("Object Thinking");
Book b2 = b1.withTitle("It");
\end{ffcode}
}
\end{pptWide}
\par
\plush{}

\print{\pptSection[Gradients]{There are four gradients of immutability}}
\print{I. Constant}
\print{II. Not a Constant}
\print{III. Represented Mutability}
\plush{VI. Encapsulated Mutability}

\pptSection[Constant]{Gradient I: Constant}
\begin{pptWide}{2}
{\small\begin{ffcode}
class Book {
  private final String t;
  Book(String t) { this.t = t; }
  String title() {
    return this.t;
  }
}
\end{ffcode}
}
\par\columnbreak\par
{\small\begin{ffcode}
Book b = new Book("Object Thinking");
String t1 = b.title();
String t2 = b.title()
\end{ffcode}
}
\end{pptWide}
\par
\plush{}

\pptSection[NotConstant]{Gradient II: Not a Constant}
\begin{pptWide}{2}
{\small\begin{ffcode}
class Book {
  private final String t;
  Book(String t) { this.t = t; }
  String title() {
    return String.format(
      "%s / %s", title, new Date()
    );
  }
}
\end{ffcode}
}
\par\columnbreak\par
{\small\begin{ffcode}
Book b = new Book("Object Thinking");
String t1 = b.title();
String t2 = b.title()
\end{ffcode}
}
\end{pptWide}
\par
\plush{}

\pptSection[NotConstant]{Gradient III: Represented Mutability}
\begin{pptWide}{2}
{\scriptsize\begin{ffcode}
class Book {
  private final Path path;
  Book(Path p) { this.path = p; }
  Book rename(String title) {
    Files.write(
      this.path,
      title.getBytes(),
      StandardOpenOption.CREATE
    );
    return this;
  }
  String title() {
    return new String(
      Files.readAllBytes(this.path)
    );
  }
}
\end{ffcode}
}
\par\columnbreak\par
{\small\begin{ffcode}
Book b = new Book("Object Thinking");
String t1 = b.title();
b.rename("Elegant Objects");
String t2 = b.title()
\end{ffcode}
}
\end{pptWide}
\par
\plush{}

\pptSection[NotConstant]{Gradient VI: Encapsulated Mutability}
\begin{pptWide}{2}
{\small\begin{ffcode}
class Book {
  private final StringBuffer buffer;
  Book rename(String t) {
    this.buffer.setLength(0);
    this.buffer.append(t);
    return this;
  }
  String title() {
    return this.buffer.toString();
  }
}
\end{ffcode}
}
\par\columnbreak\par
{\small\begin{ffcode}
Book b = new Book("Object Thinking");
String t1 = b.title();
b.rename("Elegant Objects");
String t2 = b.title()
\end{ffcode}
}
\end{pptWide}
\par
\plush{}

\plush{\pptChapter{Problems}}

\pptSection[Side-effects]{Side effects}
\begin{pptWide}{2}
With a side effect:\par
{\small\begin{ffcode}
public String post(Request request) {
  request.setMethod("POST");
  return request.fetch();
}

Request r = new Request("http://...");
r.setMethod("GET");
String first = this.post(r);
String second = r.fetch();
\end{ffcode}
}
\par\columnbreak\par
Without a side effect:\par
{\small\begin{ffcode}
public String post(Request request) {
  return request
    .withMethod("POST")
    .fetch();
}

Request r = new Request("http://...")
  .withMethod("GET");
String first = this.post(r);
String second = r.fetch();
\end{ffcode}
}
\end{pptWide}
\par
\plush{}

\pptSection[Concurrency]{Thread (un-)safety}
\begin{pptWide}{2}
{\small\begin{ffcode}
class Books {
  private int c = 0;
  void add() {
    this.c = this.s + 1;
  }
}
\end{ffcode}
\begin{tabular}{l>{\raggedright}p{11cm}}
  \raisebox{-1\height}{\pptPic{0.2}{goetz.png}} & \small Goetz et al. explained the advantages of immutable objects
  in more details in their very famous book ``Java Concurrency in Practice'' (highly recommended!) \\
\end{tabular}
}
\par\columnbreak\par
{\small\begin{ffcode}
ExecutorService e =
  Executors.newCachedThreadPool();
final Books books = new Books();
for (int i = 0; i < 1000; i++) {
  e.execute(
    new Thread(
      () -> {
        books.add();
      }
    )
  );
}
// What is the value of "books.c"?
\end{ffcode}
}
\end{pptWide}
\par
\plush{}

\pptSection[Coupling]{Temporal Coupling}
\begin{pptWide}{2}
{\small\begin{ffcode}
Request r = new Request("http://...");
r.method("POST");
String first = r.fetch();
r.body("text=hello");
String second = r.fetch();
\end{ffcode}
}
\par\columnbreak\par
{\small\begin{ffcode}
Request r = new Request("http://...");
// r.method("POST");
// String first = r.fetch();
r.body("text=hello");
String second = r.fetch();
\end{ffcode}
}
\end{pptWide}
\par
\plush{}

\pptSection[Identity]{Identity Mutability}
\begin{pptWide}{2}
{\small\begin{ffcode}
Date first = new Date(1L);
Date second = new Date(1L);
first.setTime(2L);
assert first.equals(second); // false
\end{ffcode}
}
\par\columnbreak\par
{\small\begin{ffcode}
Map<Date, String> map = new HashMap<>();
Date date = new Date();
map.put(date, "hello, world!");
date.setTime(12345L);
assert map.containsKey(date); // false
\end{ffcode}
}
\end{pptWide}
\par
\plush{}

\plush{\pptChapter{ORM}}

\pptSection[JPA]{Java Persistence API}
\begin{pptWide}{2}
{\small\begin{ffcode}
@Entity
@Table(name = "movie")
public class Movie {
  @Id
  private Long id;
  private String name;
  private Integer year;
  // ctors
  // getters
  // setters
}
\end{ffcode}
}
\par\columnbreak\par
{\small\begin{ffcode}
EntityManager em = getEntityManager();
em.getTransaction().begin();
Movie movie = new Movie();
movie.setId(1L);
movie.setName("The Godfather");
movie.setYear(1972);
em.persist(movie);
em.getTransaction().commit();
\end{ffcode}
}
\end{pptWide}
\par
\plush{}

\pptSection[SQL-speaking]{SQL speaking objects}
\begin{pptWide}{2}
{\small\begin{ffcode}
interface Movie {
  int id();
  String title();
  String author();
}
\end{ffcode}
}
\par\columnbreak\par
{\scriptsize\begin{ffcode}
final class PgMovie implements Movie
  private final Source dbase;
  private final int number;
  public PgMovie(DataSource data, int id)
    this.dbase = data;
    this.number = id;
  public String title()
    return new JdbcSession(this.dbase)
      .sql("SELECT title FROM movie WHERE id = ?")
      .set(this.number)
      .select(new SingleOutcome<String>(String.class));
  public String author()
    return new JdbcSession(this.dbase)
      .sql("SELECT name FROM movie JOIN author ON author.id = movie.author WHERE movie.id = ?")
      .set(this.number)
      .select(new SingleOutcome<String>(String.class));
\end{ffcode}
}
\end{pptWide}
\par
\plush{}

\plush{\pptChapter{Read and Watch}}

\href{https://www.yegor256.com/2014/06/09/objects-should-be-immutable.html}{Objects Should Be Immutable} by me

\href{https://www.yegor256.com/2016/09/07/gradients-of-immutability.html}{Gradients of Immutability} by me

\href{https://www.yegor256.com/2014/12/22/immutable-objects-not-dumb.html}{Immutable Objects Are Not Dumb} by me

\href{https://www.yegor256.com/2014/12/09/immutable-object-state-and-behavior.html}{How an Immutable Object Can Have State and Behavior?} by me

\href{https://www.yegor256.com/2014/11/07/how-immutability-helps.html}{How Immutability Helps} by me

\href{https://www.yegor256.com/2014/12/01/orm-offensive-anti-pattern.html}{ORM Is an Offensive Anti-Pattern} by me

\end{document}

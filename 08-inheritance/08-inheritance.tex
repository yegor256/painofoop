% (The MIT License)
%
% SPDX-FileCopyrightText: Copyright (c) 2023-2025 Yegor Bugayenko
% SPDX-License-Identifier: MIT

\documentclass{article}
\usepackage{../lecture-notes/notes}
\usetikzlibrary{arrows.meta}
\newcommand*\thetitle{Inheritance}
\newcommand*\thesubtitle{Polymorphism, Subtyping, Reuse}
\begin{document}

\lnTitlePage{8}{8}{m0OEOoGgIuM}

\pptToc

\plush{\pptChapter{Polymorphism}}

\pptSection[LSP]{Liskov Substitution Principle}
\lnQuote
  {barbara-liskov}
  {If for each object $o_1$ of type $S$ there is an object $o_2$ of type $T$ such that for all programs $P$ defined in terms of $T$, the behavior of $P$ is unchanged when $o_1$ is substituted for $o_2$, then $S$ is a subtype of $T$.}
  {liskov1987keynote}

\pptSection[SOLID]{SOLID (the ``L'' part)}
\lnQuote
  {robert-martin}
  {Functions that use pointers or references to base classes must be able to use objects of derived classes without knowing it.}
  {martin2008clean}

\pptSection[Subtyping]{Subtyping}
\begin{pptWide}{2}
{\small\begin{ffcode}
interface Figure
  float area();

interface Circle extends Figure
  float perimeter();

interface Polygon extends Figure
  int sides();

void paint(Figure f)
  float s = f.area();
  // ...
\end{ffcode}
}
\par\columnbreak\par
\begin{tikzpicture}[
  every path/.style={line width=3pt,-{Stealth[inset=0pt,width=10mm,length=10mm,open]}},
  every node/.style={font={\ttfamily}}]
\node[draw] (figure) {Figure};
\node[draw, below right=4cm and 2cm of figure] (circle) {Circle};
\node[draw, below left=3cm and 2cm of figure] (polygon) {Polygon};
\draw (circle) -- (figure);
\draw (polygon) -- (figure);
\end{tikzpicture}
\par
$ \texttt{Circle} \sqsubseteq \texttt{Figure} $
\par
\texttt{Circle <: Figure}
\end{pptWide}
\plush{}

\pptSection[Generics]{Parametric Polymorphism (Generics)}
\begin{pptWide}{2}
{\small\begin{ffcode}
class StackOfStrings {
  void push(String str) // ...
  String pop() // ...

class StackOfIntegers {
  void push(Integer num) // ...
  Integer pop() // ...

var s1 = new StackOfStrings();
s1.push("Hello, world!");

var s2 = new StackOfIntegers();
s2.push(42);
\end{ffcode}
}
\par\columnbreak\par
{\small\begin{ffcode}
class <T> Stack<T> {
  void push(T item) // ...
  T pop() // ...
}

var s1 = new Stack<String>();
s1.push("Hello, world!");

var s2 = new Stack<Integer>();
s2.push(42);
\end{ffcode}
}
\end{pptWide}
\plush{}

\pptSection[Overloading]{Ad Hoc Polymorphism (Method Overloading)}
\begin{pptWide}{2}
{\small\begin{ffcode}
class Cart {
  void add(int pid) // ...
  void addString(String pid) {
    this.add(Integer.parseInt(pid));
  }
}

var c = new Cart();
c.add(42);
c.addString("17");
c.addString("Hello, world!");
\end{ffcode}
}
\par\columnbreak\par
{\small\begin{ffcode}
class Cart {
  void add(int pid) // ...
  void add(String pid) {
    this.add(Integer.parseInt(pid));
  }
}

var c = new Cart();
c.add(42);
c.add("17");
c.add("Hello, world!");
\end{ffcode}
}
\end{pptWide}
\plush{}

\plush{\pptChapter[Inheritance]{Implementation Inheritance}}

\lnQuote
  [Grady Booch]
  {grady-booch}
  {However, there is tension between the concepts of coupling and inheritance because inheritance introduces significant coupling. On the one hand, weakly coupled classes are desirable; on the other hand, inheritance---which \ul{tightly couples} superclasses and their subclasses---helps us to exploit the commonality among abstractions.}
  {booch1994object}

\lnQuote
  {allen-holub}
  {The |extends| keyword is evil; maybe not at the Charles Manson level, but bad enough that it should be shunned whenever possible.}
  {holub2003extends}

\lnQuote
  {james-gosling}
  {Someone asked him: ``If you could do Java over again, what would you change?'' ``I'd leave out classes,'' he replied.}
  {holub2003extends}

\pptSection[Reuse]{Code reuse}
\begin{pptWide}{2}
{\small\begin{ffcode}
class Square
  private float width;
  float area()
    return width * width;

class Circle extends Square
  Circle(float radius)
    super(radius);
  @Override float area()
    return 3.14 * super.area();
\end{ffcode}
}
\par\columnbreak\par
\begin{tikzpicture}[
  every path/.style={line width=3pt,-{Stealth[inset=0pt,width=10mm,length=10mm,open]}},
  every node/.style={font={\ttfamily}}]
\node[draw] (square) {Square};
\node[draw, below right=4cm and 2cm of square] (circle) {Circle};
\draw[draw=red] (circle) -- (square);
\end{tikzpicture}
\par
Here, the |Circle| is \ul{not} a |Square|.
It merely reuses the code that was negligently left open in the |Square|.
\end{pptWide}
\plush{}

\plush{\pptThought{Inheriting means ``receive (money, property, or a title) as an heir at the death of the previous holder.'' Who is dead, you ask? An object is dead if it allows other objects to inherit its encapsulated code and data.}}

\pptSection[Composition]{Composition over inheritance}
\begin{pptWide}{2}
Implementation Inheritance:
\par
{\small\begin{ffcode}
class Square
  private float width;
  float area()
    return width * width;

class Circle extends Square
  Circle(float radius)
    super(radius);
  @Override float area()
    return 3.14 * super.area();
\end{ffcode}
}
\par\columnbreak\par
Composition:
\par
{\small\begin{ffcode}
final class Square
  private float width;
  float area()
    return width * width;

final class Circle
  private Square s;
  Circle(float radius)
    this.s = new Square(radius);
  float area()
    return 3.14 * s.area();
\end{ffcode}
}
\end{pptWide}
\plush{}

\plush{\pptThought{All classes, without exceptions, should be either \ff{final} or \ff{abstract}}}

\pptSection[Multiple]{Multiple inheritance}
\begin{pptWide}{2}
\par
{\small\begin{ffcode}
class Pi
  float value()
    return 3.1415926;

class Square
  private float width;
  float area()
    return width * width;

class Circle extends Square, Pi
  Circle(float r): Square(r), Pi() {}
  virtual float area()
    return Pi.value() * Square.area();
\end{ffcode}
}
\par\columnbreak\par
\begin{tikzpicture}[
  every path/.style={line width=3pt,-{Stealth[inset=0pt,width=10mm,length=10mm,open]}},
  every node/.style={font={\ttfamily}}]
\node[draw] (square) {Square};
\node[draw, below right=0.5cm and 6cm of square] (pi) {Pi};
\node[draw, below right=4cm and 2cm of square] (circle) {Circle};
\draw[draw=red] (circle) -- (square);
\draw[draw=red] (circle) -- (pi);
\end{tikzpicture}
\end{pptWide}
\plush{}

\pptSection[Parents]{Multiple super types}
\begin{pptWide}{2}
\par
{\small\begin{ffcode}
interface Actor
  void move(int dx, int dy);

interface Figure
  float area();

class Circle implements Figure, Actor
  Circle(float r)
  @Override float area()
    // ...
  @Override void move(int dx, int dy)
    // ...
\end{ffcode}
}
\par\columnbreak\par
\begin{tikzpicture}[
  every path/.style={line width=3pt,-{Stealth[inset=0pt,width=10mm,length=10mm,open]}},
  every node/.style={font={\ttfamily}}]
\node[draw] (figure) {Figure};
\node[draw, below right=0.5cm and 6cm of figure] (actor) {Actor};
\node[draw, below right=4cm and 2cm of figure] (circle) {Circle};
\draw[draw=green] (circle) -- (figure);
\draw[draw=green] (circle) -- (actor);
\end{tikzpicture}
\end{pptWide}
\plush{}

\end{document}

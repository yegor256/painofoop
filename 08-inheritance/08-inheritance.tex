% (The MIT License)
%
% Copyright (c) 2023 Yegor Bugayenko
%
% Permission is hereby granted, free of charge, to any person obtaining a copy
% of this software and associated documentation files (the 'Software'), to deal
% in the Software without restriction, including without limitation the rights
% to use, copy, modify, merge, publish, distribute, sublicense, and/or sell
% copies of the Software, and to permit persons to whom the Software is
% furnished to do so, subject to the following conditions:
%
% The above copyright notice and this permission notice shall be included in all
% copies or substantial portions of the Software.
%
% THE SOFTWARE IS PROVIDED 'AS IS', WITHOUT WARRANTY OF ANY KIND, EXPRESS OR
% IMPLIED, INCLUDING BUT NOT LIMITED TO THE WARRANTIES OF MERCHANTABILITY,
% FITNESS FOR A PARTICULAR PURPOSE AND NONINFRINGEMENT. IN NO EVENT SHALL THE
% AUTHORS OR COPYRIGHT HOLDERS BE LIABLE FOR ANY CLAIM, DAMAGES OR OTHER
% LIABILITY, WHETHER IN AN ACTION OF CONTRACT, TORT OR OTHERWISE, ARISING FROM,
% OUT OF OR IN CONNECTION WITH THE SOFTWARE OR THE USE OR OTHER DEALINGS IN THE
% SOFTWARE.

\documentclass{article}
\usepackage{../painofoop}
\usetikzlibrary{arrows.meta}
\newcommand*\thetitle{Inheritance}
\newcommand*\thesubtitle{...}
\begin{document}

\plush{\poopTitlePage{8}}

\pptToc

\plush{\pptChapter{Polymorphism}}

\pptSection[LSP]{Liskov Substitution Principle}
\plush{\begin{pptMiddle}\pptQuote{barbara-liskov.jpg}{If for each object $o_1$ of type $S$ there is an object $o_2$ of type $T$ such that for all programs $P$ defined in terms of $T$, the behavior of $P$ is unchanged when $o_1$ is substituted for $o_2$, then $S$ is a subtype of $T$}{Barbara Liskov, \emph{Keynote Address: Data Abstraction and Hierarchy}, Addendum to the Proceedings on Object-oriented Programming Systems, Languages and Applications, 1987}\end{pptMiddle}}

\pptSection[SOLID]{SOLID (the ``L'' part)}
\plush{\begin{pptMiddle}\pptQuote{robert-martin.jpg}{Functions that use pointers or references to base classes must be able to use objects of derived classes without knowing it}{Robert C. Martin, \emph{Design Principles and Design Patterns discussing}, 2000}\end{pptMiddle}}

\pptSection[Subtyping]{Subtyping}
\begin{pptWide}{2}
{\small\begin{ffcode}
interface Figure
  float surface();

interface Circle extends Figure
  float perimeter();

interface Polygon extends Figure
  int sides();

void paint(Figure f)
  float s = f.surface();
  // ...
\end{ffcode}
}
\par\columnbreak\par
\begin{tikzpicture}[
  every path/.style={line width=3pt,-{Stealth[inset=0pt,width=10mm,length=10mm,open]}},
  every node/.style={font={\ttfamily}}]
\node[draw] (figure) {Figure};
\node[draw, below right=4cm and 2cm of figure] (circle) {Circle};
\node[draw, below left=3cm and 2cm of figure] (polygon) {Polygon};
\draw (circle) -- (figure);
\draw (polygon) -- (figure);
\end{tikzpicture}
\par
$ \texttt{Circle} \sqsubseteq \texttt{Figure} $
\par
\texttt{Circle <: Figure}
\end{pptWide}
\plush{}

\pptSection[Generics]{Parametric Polymorphism (Generics)}
\begin{pptWide}{2}
{\small\begin{ffcode}
class StackOfStrings {
  push(String str) // ...
  String pop() // ...

class StackOfIntegers {
  push(Integer num) // ...
  Integer pop() // ...

var s1 = new StackOfStrings();
s1.push("Hello, world!");

var s2 = new StackOfIntegers();
s2.push(42);
\end{ffcode}
}
\par\columnbreak\par
{\small\begin{ffcode}
class <T> Stack<T> {
  push(T item) // ...
  T pop() // ...
}

var s1 = new Stack<String>();
s1.push("Hello, world!");

var s2 = new Stack<Integer>();
s2.push(42);
\end{ffcode}
}
\end{pptWide}
\plush{}

\pptSection[Overloading]{Ad Hoc Polymorphism (Method Overloading)}
\begin{pptWide}{2}
{\small\begin{ffcode}
class Cart {
  void add(int pid) // ...
  void addString(String pid) {
    this.add(Integer.parseInt(pid));
  }
}

var c = new Cart();
c.add(42);
c.addString("17");
c.addString("Hello, world!");
\end{ffcode}
}
\par\columnbreak\par
{\small\begin{ffcode}
class Cart {
  void add(int pid) // ...
  void add(String pid) {
    this.add(Integer.parseInt(pid));
  }
}

var c = new Cart();
c.add(42);
c.add("17");
c.add("Hello, world!");
\end{ffcode}
}
\end{pptWide}
\plush{}

\plush{\pptChapter[Inheritance]{Implementation Inheritance}}

\plush{\pptThought{Inheriting means ``receive (money, property, or a title) as an heir at the death of the previous holder.'' Who is dead, you ask? An object is dead if it allows other objects to inherit its encapsulated code and data.}}

\pptSection[Reuse]{Code reuse}
\begin{pptWide}{2}
{\small\begin{ffcode}
class Square
  private float width;
  float surface()
    return width * width;

class Circle extends Square
  Circle(float radius)
    super(radius);
  @Override float surface()
    return 3.14 * super.surface();
\end{ffcode}
}
\par\columnbreak\par
\begin{tikzpicture}[
  every path/.style={line width=3pt,-{Stealth[inset=0pt,width=10mm,length=10mm,open]}},
  every node/.style={font={\ttfamily}}]
\node[draw] (square) {Square};
\node[draw, below right=4cm and 2cm of square] (circle) {Circle};
\draw[draw=red] (circle) -- (square);
\end{tikzpicture}
\par
Here, the |Circle| is \emph{not} a |Square|.
It merely reuses the code that was negligently left open in the |Square|.
\end{pptWide}
\plush{}

\pptSection[Composition]{Composition over inheritance}
\begin{pptWide}{2}
Implementation Inheritance:
\par
{\small\begin{ffcode}
class Square
  private float width;
  float surface()
    return width * width;

class Circle extends Square
  Circle(float radius)
    super(radius);
  @Override float surface()
    return 3.14 * super.surface();
\end{ffcode}
}
\par\columnbreak\par
Composition:
\par
{\small\begin{ffcode}
final class Square
  private float width;
  float surface()
    return width * width;

final class Circle
  private Square s;
  Circle(float radius)
    this.s = new Square(radius);
  float surface()
    return 3.14 * s.surface();
\end{ffcode}
}
\end{pptWide}
\plush{}


final classes -- not welcome to extend

code reuse via composition vs reuse via inheritance

two meanings of the word "inherit", per the dictionary

\plush{\pptChapter[Quiz]{Quiz}}

\plush{\begin{pptMiddle}
I show this code to job interview candidates, asking them to find as many defects
in it as they can: \href{https://github.com/yegor256/quiz/blob/master/Parser.java}{yegor256/quiz}~(Java).
\par
How many problems you can find in this code?
\end{pptMiddle}}

\plush{\pptChapter[R\&W]{Read and Watch}}

\href{https://www.infoworld.com/article/2073649/why-extends-is-evil.html}{Why extends is evil} by Allen Holub (2003)

\href{https://www.yegor256.com/2016/09/13/inheritance-is-procedural.html}{Inheritance Is a Procedural Technique for Code Reuse} by me (2016)

\href{https://www.youtube.com/watch?v=DjrA7_Uymok}{Inheritance vs. Subtyping (Webinar \#24)} by me (2017)
\end{document}

% (The MIT License)
%
% Copyright (c) 2023 Yegor Bugayenko
%
% Permission is hereby granted, free of charge, to any person obtaining a copy
% of this software and associated documentation files (the 'Software'), to deal
% in the Software without restriction, including without limitation the rights
% to use, copy, modify, merge, publish, distribute, sublicense, and/or sell
% copies of the Software, and to permit persons to whom the Software is
% furnished to do so, subject to the following conditions:
%
% The above copyright notice and this permission notice shall be included in all
% copies or substantial portions of the Software.
%
% THE SOFTWARE IS PROVIDED 'AS IS', WITHOUT WARRANTY OF ANY KIND, EXPRESS OR
% IMPLIED, INCLUDING BUT NOT LIMITED TO THE WARRANTIES OF MERCHANTABILITY,
% FITNESS FOR A PARTICULAR PURPOSE AND NONINFRINGEMENT. IN NO EVENT SHALL THE
% AUTHORS OR COPYRIGHT HOLDERS BE LIABLE FOR ANY CLAIM, DAMAGES OR OTHER
% LIABILITY, WHETHER IN AN ACTION OF CONTRACT, TORT OR OTHERWISE, ARISING FROM,
% OUT OF OR IN CONNECTION WITH THE SOFTWARE OR THE USE OR OTHER DEALINGS IN THE
% SOFTWARE.

\documentclass{article}
\usepackage{../painofoop}
\usetikzlibrary{arrows.meta}
\newcommand*\thetitle{Inheritance}
\newcommand*\thesubtitle{...}
\begin{document}

\plush{\poopTitlePage{8}}

\pptToc

\plush{\pptChapter[Subtyping]{Subtyping}}

\plush{\begin{pptMiddle}\pptQuote{barbara-liskov.jpg}{If for each object $o_1$ of type $S$ there is an object $o_2$ of type $T$ such that for all programs $P$ defined in terms of $T$, the behavior of $P$ is unchanged when $o_1$ is substituted for $o_2$, then $S$ is a subtype of $T$}{Barbara Liskov, \emph{Keynote Address: Data Abstraction and Hierarchy}, Addendum to the Proceedings on Object-oriented Programming Systems, Languages and Applications, 1987}\end{pptMiddle}}

\pptSection{Subtyping (Polymorphism)}
\begin{pptWide}{2}
{\small\begin{ffcode}
interface Figure
  float surface();

interface Circle extends Figure
  float perimeter();

interface Polygon extends Figure
  int sides();

void paint(Figure f)
  float s = f.surface();
  // ...
\end{ffcode}
}
\par\columnbreak\par
\begin{tikzpicture}[
  every path/.style={line width=3pt,-{Stealth[inset=0pt,width=10mm,length=10mm,open]}},
  every node/.style={font={\ttfamily}}]
\node[draw] (figure) {Figure};
\node[draw, below right=4cm and 2cm of figure] (circle) {Circle};
\node[draw, below left=3cm and 2cm of figure] (polygon) {Polygon};
\draw (circle) -- (figure);
\draw (polygon) -- (figure);
\end{tikzpicture}
\par
$ \texttt{Circle} \sqsubseteq \texttt{Figure} $
\par
\texttt{Circle <: Figure}
\end{pptWide}
\plush{}

\pptSection{Parametric Polymorphism (Generics)}
\begin{pptWide}{2}
{\small\begin{ffcode}
class StackOfStrings {
  push(String str) // ...
  String pop() // ...

class StackOfIntegers {
  push(Integer num) // ...
  Integer pop() // ...

var s1 = new StackOfStrings();
s1.push("Hello, world!");

var s2 = new StackOfIntegers();
s2.push(42);
\end{ffcode}
}
\par\columnbreak\par
{\small\begin{ffcode}
class <T> Stack<T> {
  push(T item) // ...
  T pop() // ...
}

var s1 = new Stack<String>();
s1.push("Hello, world!");

var s2 = new Stack<Integer>();
s2.push(42);
\end{ffcode}
}
\end{pptWide}
\plush{}

implementation inheritance vs subtyping

final classes -- not welcome to extend

code reuse via composition vs reuse via inheritance

liskov substitution principle

two meanings of the word "inherit", per the dictionary

\plush{\pptThought{Inheriting means ``receive (money, property, or a title) as an heir at the death of the previous holder.'' Who is dead, you ask? An object is dead if it allows other objects to inherit its encapsulated code and data.}}

\plush{\pptChapter[R\&W]{Read and Watch}}

\href{https://www.infoworld.com/article/2073649/why-extends-is-evil.html}{Why extends is evil} by Allen Holub (2003)

\href{https://www.yegor256.com/2016/09/13/inheritance-is-procedural.html}{Inheritance Is a Procedural Technique for Code Reuse} by me (2016)

\href{https://www.youtube.com/watch?v=DjrA7_Uymok}{Inheritance vs. Subtyping (Webinar \#24)} by me (2017)
\end{document}

% SPDX-FileCopyrightText: Copyright (c) 2023-2026 Yegor Bugayenko
% SPDX-License-Identifier: MIT

\documentclass{article}
\usepackage{../lecture-notes/notes}
\newcommand*\thetitle{NULL}
\newcommand*\thesubtitle{Fail Fast, Returning, Checking, Objects}
\begin{document}

\lnTitlePage{6}{8}{cTyIF20Kmz4}

\lnQuote
  {tony-hoare}
  {I was designing the first comprehensive type system for references in an OO language (ALGOL~W). My goal was to ensure that all use of references should be absolutely safe. But I couldn't resist the temptation to put in a null reference, simply because it was so easy to implement. This has led to innumerable errors, vulnerabilities, and system crashes, which have probably caused a billion dollars of \ul{pain} and \ul{damage} in the last forty years.}
  {hoare2009null}

\pptToc

\plush{\pptChapter[FailFast]{Fail Fast vs Fail Safe}}

\lnQuote
  {james-shore}
  {Over time, more and more errors will \ul{fail fast}, and you'll see the cost of debugging decrease and the \ul{quality} of your system improve.}
  {shore2004fail}

\begin{lnSnippet}[defaults-safe.java]
int size(File file) {
  if (!file.exists()) {
    return 0;
  }
  return file.length();
}
\end{lnSnippet}
\begin{lnSnippet}[defaults-fast.java]
int size(File file) {
  if (!file.exists()) {
    throw new IllegalArgumentException(
      "The file is absent :("
    );
  }
  return file.length();
}
\end{lnSnippet}
\lnPitch{
  \pptSection{Defaults}
  \begin{pptWide}{2}
  Fail Safe:\par
  {\small\ffinput{defaults-safe.java}}
  \par\columnbreak\par
  Fail Fast:\par
  {\small\ffinput{defaults-fast.java}}
  \end{pptWide}
  \par
  The right snippet is more \ul{fragile}, leading to more errors in runtime, but eventually ... leading to less bugs~\citep{bugayenko2015blog0825}.}

\begin{lnSnippet}[swallowing-bad.java]
String read(File file) {
  try {
    return new String(
      Files.readBytes(file)
    );
  } catch (IOException e) {
    e.printStackTrace();
    return ""; // default
  }
}
\end{lnSnippet}
\begin{lnSnippet}[swallowing-good.java]
String read(File file) {
  try {
    return new String(
      Files.readBytes(file));
  } catch (IOException e) {
    throw new IllegalStateException(
      String.format(
        "Can't read file %s", e.name()),
      e);
  }
}
\end{lnSnippet}
\lnPitch{
  \pptSection[Swallowing]{Exception swallowing}
  \begin{pptWide}{2}
  {\small\ffinput{swallowing-bad.java}}
  \par\columnbreak\par
  {\small\ffinput{swallowing-good.java}}
  \end{pptWide}
  \par
  The right snippet is \ul{escalating}, while the left one is \ul{swallowing}.}

\plush{
  \pptSection[SDLC]{Software Development Lifecycle}
  \par
  \begin{tikzpicture}[every path/.style={->, draw, color=gray, line width=5pt}, every node/.style={font=\large, color=black}]
  \node[] (fix) {Fix};
  \node[above right=4cm of fix] (deploy) {Deploy};
  \node[right=9cm of fix] (use) {Use};
  \node[below right=4cm of fix] (report) {Report};
  \path (fix) -- (deploy);
  \path (deploy) -- (use);
  \path (use) -- (report);
  \path (report) -- (fix);
  \path[dashed] (use) -- (fix);
  \end{tikzpicture}
  \par
  Watch this video from DEVit'2016 conference: \\
  \href{https://www.youtube.com/watch?v=nCGBgI1MNwE}{Need It Robust?Make It Fragile!}}

% \plush{
%   \pptSection[Complexity]{Is There a Correlaction Between NULL and CC?}
%   \begin{multicols}{2}
%   \includegraphics[width=\linewidth]{null-vs-cc.png}
%   \par\columnbreak\par
%   \tbd{...}
%   \par
%   \lnSource{garmash2021exploring}
%   \end{multicols}}

\plush{\pptChapter[Returning]{Alternatives to Returning NULL}}

\begin{lnSnippet}[return-null.java]
String nameOfEmployee(int id) {
  if (!em.existsInDb(id)) {
    return null;
  }
  return em.readFromDb(id);
}
\end{lnSnippet}
\begin{lnSnippet}[return-exception.java]
String nameOfEmployee(int id) {
  if (em.existsInDb(id)) {
    throw new EmployeeNotFound(id);
  }
  return em.readFromDb(id);
}
\end{lnSnippet}
\lnPitch{
  \pptSection[Return]{Returning NULL or raising an error?}
  \begin{pptWide}{2}
  {\small\ffinput{return-null.java}}
  \par\columnbreak\par
  {\small\ffinput{return-exception.java}}
  \end{pptWide}
  \par
  The right snippet is ``Fail Fast,'' that's why more preferable~\citep{bugayenko2014blog0513, bugayenko2015blog1201}.}

\begin{lnSnippet}[list-null.java]
String nameOfEmployee(int id) {
  if (!em.existsInDb(id)) {
    return null;
  }
  return em.readFromDb(id);
}
\end{lnSnippet}
\begin{lnSnippet}[list-collection.java]
List<String> nameOfEmployee(int id) {
  List<String> names =
    new ArrayList<>(0);
  if (em.existsInDb(id)) {
    names.add(em.readFromDb(id));
  }
  return names;
}
\end{lnSnippet}
\lnPitch{
  \pptSection[List]{Returning a List or a NULL?}
  \begin{pptWide}{2}
  {\small\ffinput{list-null.java}}
  \par\columnbreak\par
  {\small\ffinput{list-collection.java}}
  \end{pptWide}
  \par
  There are mo elegant alternatives in most languages, like |Optional| in Java~8+~\citep{bugayenko2018blog0522}.}

\plush{
  \begin{pptMiddle}
  \pptPic{0.8}{tweet-vote.png}
  \end{pptMiddle}}

\begin{lnSnippet}[fake-null.java]
Employee employee(int id) {
  if (!em.existsInDb(id)) {
    return null;
  }
  return new PgEmployee(id);
}

e = employee(42);
print(e.id());
print(e.salary());
\end{lnSnippet}
\begin{lnSnippet}[fake-entity.java]
Employee employee(int id) {
  if (!em.existsInDb(id)) {
    return FakeEmployee(id);
  }
  return new PgEmployee(id);
}

e = employee(42);
print(e.id());

print(e.salary());
\end{lnSnippet}
\lnPitch{
  \pptSection[Fake]{Returning a Fake Entity}
  \begin{pptWide}{2}
  {\small\ffinput{fake-null.java}}
  \par\columnbreak\par
  {\small\ffinput{fake-entity.java}}
  \end{pptWide}}

\plush{\pptChapter[Checking]{Alternatives to Checking for NULL}}

\begin{lnSnippet}[coalescing-manual.cs]
int? sizeOf(File f) {
  if (!f.exists()) {
    return null;
  }
  return f.size();
}

int? s = sizeOf(f);
if (s == null) {
  s = 0;
}
\end{lnSnippet}
\begin{lnSnippet}[coalescing-operator.cs]
int? sizeOf(File f) {
  if (!f.exists()) {
    return null;
  }
  return f.size();
}

int s = sizeOf(f) ?? 0;
\end{lnSnippet}
\lnPitch{
  \pptSection[??-operator]{null-coalescing operator in C\#}
  \begin{pptWide}{2}
  {\small\ffinput{coalescing-manual.cs}}
  \par\columnbreak\par
  {\small\ffinput{coalescing-operator.cs}}
  \end{pptWide}
  \par
  Both snippets are bad design, though. They are workarounds.}

\begin{lnSnippet}[ruby-manual.rb]
def employee(id)
  unless db.exists?(id)
    return nil
  end
  return db.get(id)
end

e = employee(42)
puts e.name unless e.nil?
\end{lnSnippet}
\begin{lnSnippet}[ruby-operator.rb]
def employee(id)
  unless db.exists?(id)
    return nil
  end
  return db.get(id)
end

puts employee(42)&.name
\end{lnSnippet}
\lnPitch{
  \pptSection[Ruby]{\texttt{\&.} operator in Ruby}
  \begin{pptWide}{2}
  {\small\ffinput{ruby-manual.rb}}
  \par\columnbreak\par
  {\small\ffinput{ruby-operator.rb}}
  \end{pptWide}
  \par
  Actually, the snippets produce different output when the employee is not found. How are they different?}

\begin{lnSnippet}[kotlin-null.kt]
var a: String = "abc"
a = null // compilation error

var b: String? = "abc"
b = null // no error here

println(b?.length) // prints what?
println(b?.length ?: -1) // Elvis operator
\end{lnSnippet}
\lnPitch{
  \pptSection[Kotlin]{NULL-awareness in Kotlin}
  \pptPin{\pptPic{0.9}{elvis-operator.jpg}}
  {\small\ffinput{kotlin-null.kt}}}

\plush{\pptChapter[Checking]{Alternatives to Storing NULL}}

\begin{lnSnippet}[immutable-mutable.java]
class Employee {
  private String name = null;
  void setName(String n) {
    this.name = n;
  }
}

e = new Employee();
e.setName("Jeff");
\end{lnSnippet}
\begin{lnSnippet}[immutable-immutable.java]
class Employee {
  private final String name;
  Employee(String n) {
    this.name = n;
  }
  Employee withName(String n) {
    return new Employee(n);
  }
}

e1 = new Employee();
e2 = e1.withName("Jeff");
\end{lnSnippet}
\lnPitch{
  \pptSection[Immutability]{Immutable objects}
  \begin{pptWide}{2}
  {\small\ffinput{immutable-mutable.java}}
  \par\columnbreak\par
  {\small\ffinput{immutable-immutable.java}}
  \end{pptWide}}

\plush{\pptChapter[Objects]{Object Thinking}}

\begin{lnSnippet}[thinking-code.java]
d = getDepartment(42);
e = d.getEmployee("Jeff");
if (e != null) {
  printf("Hello, %s", e.name());
}
\end{lnSnippet}
\begin{lnSnippet}[thinking-conversation.txt]
- Hello, is it the department no.42?
- Yes.
- Let me talk to your employee "Jeff".
- Hold the line please...
- Hello.
- Are you NULL?
\end{lnSnippet}
\lnPitch{
  \pptBanner{Pay respect to your objects!}
  \begin{pptWide}{2}
  {\small\ffinput{thinking-code.java}}
  \par\columnbreak\par
  {\small\ffinput{thinking-conversation.txt}}
  \includegraphics[width=2.2in]{oops.jpg}
  \end{pptWide}}

\plush{\pptChapter[Spring]{NULLs in Spring Boot}}

\lnPitch{
  \pptPic{0.8}{tweet.png}
  \par
  You can do you own analysis of existing Java open source GitHub repositories to see how often their developers use |null| keyword.
  \par
  The Takes framework is here: \href{https://github.com/yegor256/takes}{yegor256/takes}.}

\end{document}

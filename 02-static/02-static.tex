% (The MIT License)
%
% Copyright (c) 2023 Yegor Bugayenko
%
% Permission is hereby granted, free of charge, to any person obtaining a copy
% of this software and associated documentation files (the 'Software'), to deal
% in the Software without restriction, including without limitation the rights
% to use, copy, modify, merge, publish, distribute, sublicense, and/or sell
% copies of the Software, and to permit persons to whom the Software is
% furnished to do so, subject to the following conditions:
%
% The above copyright notice and this permission notice shall be included in all
% copies or substantial portions of the Software.
%
% THE SOFTWARE IS PROVIDED 'AS IS', WITHOUT WARRANTY OF ANY KIND, EXPRESS OR
% IMPLIED, INCLUDING BUT NOT LIMITED TO THE WARRANTIES OF MERCHANTABILITY,
% FITNESS FOR A PARTICULAR PURPOSE AND NONINFRINGEMENT. IN NO EVENT SHALL THE
% AUTHORS OR COPYRIGHT HOLDERS BE LIABLE FOR ANY CLAIM, DAMAGES OR OTHER
% LIABILITY, WHETHER IN AN ACTION OF CONTRACT, TORT OR OTHERWISE, ARISING FROM,
% OUT OF OR IN CONNECTION WITH THE SOFTWARE OR THE USE OR OTHER DEALINGS IN THE
% SOFTWARE.

\documentclass{article}
\usepackage{../painofoop}
\newcommand*\thetitle{Static}
\newcommand*\thesubtitle{Methods, Attributes, FastJson}
\begin{document}

\plush{\poopTitlePage{2}}

\pptToc

\plush{\pptChapter{Methods}}

\pptSection[Purpose]{What static methods are for?}
\begin{pptWide}{2}
{\small\begin{ffcode}
class Circle {
  public float radius;
}
class GeometryUtils {
  static float calcSquare(Circle c) {
    return c.radius * c.radius * 3.14;
  }
}
\end{ffcode}
}
\par\columnbreak\par
{\small\begin{ffcode}
class Circle {
  public float radius;
  float square() {
    return radius * radius * 3.14;
  }
}
\end{ffcode}
}
\end{pptWide}\par
Most notable Java examples:
FileUtils, IOUtils, and StringUtils from Apache Commons;
Files from JDK7;
Iterators from Google Guava.
Read \href{https://www.yegor256.com/2015/02/20/utility-classes-vs-functional-programming.html}{this}.
\plush{}

\print{\pptSection[Problems]{What's wrong with ``Utils''?}}
\print{1) They are \emph{unbreakable} dependencies}
\print{2) They are \emph{eager}, not lazy}
\plush{3) They are \emph{not cohesive}}

\pptSection[Coupling]{Tight Coupling}
\begin{pptWide}{2}
{\small\begin{ffcode}
void paintIt(Circle c) {
  float s = GeometryUtils.calcSquare(c);
  float p = s * 5.55;
  // paint it using the "p"
}
\end{ffcode}
}
\par\columnbreak\par
{\small\begin{ffcode}
void paintIt(Circle c) {
  float s = c.square();
  float p = s * 5.55;
  // paint it using the "p"
}
\end{ffcode}
}
\end{pptWide}\par
Which snippet is easier to test? Try to write a test for the first one, expecting \ff{s} to be equal to \ff{42.0}.
Read \href{https://www.yegor256.com/2014/05/05/oop-alternative-to-utility-classes.html}{this}.
\plush{}

\pptSection[Eagerness]{Imperative, not Declarative}
\begin{pptWide}{2}
{\small\begin{ffcode}
void paintIt(Circle c) {
  float s = GeometryUtils.calcSquare(c);
  if (t) { return; }
  float p = s * 5.55;
  // paint it using the "p"
}
\end{ffcode}
}
\par\columnbreak\par
{\small\begin{ffcode}
void paintIt(Circle c) {
  float s = new SquareOf(c);
  if (t) { return; }
  float p = s * 5.55;
  // paint it using the "p"
}
\end{ffcode}
}
\end{pptWide}\par
Which snippet is more eager to calculate the square of the circle? Which one does it when it's \emph{really} necessary?
Read \href{https://www.yegor256.com/2015/02/26/composable-decorators.html}{this}.
\plush{}

\pptSection[Cohesion]{Low Cohesion}
\begin{pptWide}{2}
{\small\begin{ffcode}
class GeometryUtils {
  static float calcSquare(Circle c);
  static float calcPerimeter(Circle c);
  static float calcSinus(Angle a);
  static float calcCosinus(float s);
  // and many more...
}
\end{ffcode}
}
\par\columnbreak\par
{\small\begin{ffcode}
class Circle {
  float square();
  float perimeter();
}
class Angle {
  float sinus();
}
class Float {
  float cosinus();
}
\end{ffcode}
}
\end{pptWide}\par
Which class looks more cohesive to you, the utility class \ff{GeometryUtils} or the \ff{Circle}?
\plush{}

\plush{\pptChapter{Attributes}}

\pptSection[Literals]{Public literals}
\begin{pptWide}{2}
{\small\begin{ffcode}
class Constants {
  public static float PI = 3.1415926;
  public static String UTF_8 = "utf-8";
  public static String LOCALE = "fr";
  // and many more
}
println("S'il vous plaît",
  Constants.LOCALE);
printf("It is %see speech!",
  Constants.LOCALE);
\end{ffcode}
}
\par\columnbreak\par
{\small\begin{ffcode}
class Print { }
class TextInFrench { }

new Print(
  new TextInFrench(
    "S'il vous plaît"
  )
)
\end{ffcode}
}
\end{pptWide}\par
We must solve the problem of functionality duplication, not just data duplication.
Read \href{https://www.yegor256.com/2015/07/06/public-static-literals.html}{this}.
\plush{}

\pptSection[Singletons]{Singletons}
\begin{pptWide}{2}
{\small\begin{ffcode}
class Canvas {
  public static Canvas INSTANCE =
    new Canvas();
  private Canvas() {}
  public void addCircle(Circle c);
}

Canvas.INSTANCE.addCircle(c1);
Canvas.INSTANCE.addCircle(c2);
\end{ffcode}
}
\par\columnbreak\par
{\small\begin{ffcode}
c = new Canvas();
c.addCircle(c1);
c.addCircle(c2);
\end{ffcode}
}
\end{pptWide}\par
Read \href{https://www.yegor256.com/2016/06/27/singletons-must-die.html}{this}.
\plush{}

\plush{\pptChapter{FastJson as Example}}

\end{document}

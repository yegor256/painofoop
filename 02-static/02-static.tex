% SPDX-FileCopyrightText: Copyright (c) 2023-2025 Yegor Bugayenko
% SPDX-License-Identifier: MIT

\documentclass{article}
\usepackage{../lecture-notes/notes}
\newcommand*\thetitle{Static}
\newcommand*\thesubtitle{Methods, Attributes, FastJson}
\begin{document}

\lnTitlePage{2}{8}{lELJSj9mWbI}

\pptToc

\plush{\pptChapter{Pre-Test}}

\begin{lnSnippet}[entities.java]
class FigureUtils {
  static final float PI = 3.1415926;
  static final FigureUtils INSTANCE;
  static {
    INSTANCE = new FigureUtils();
  }
  private FigureUtils() { /* empty */ }
  static float perimeter(Circle c) {
    return 2 * c.radius * PI;
  }
}
\end{lnSnippet}
\lnPitch{
  \pptSection[Entities]{How many static ``entities'' do you see here?}
  \begin{pptWide}{2}
  {\small\ffinput{entities.java}}
  \par\columnbreak\par
  How many?
  \begin{itemize}
  \item three
  \item four
  \item five
  \item six
  \item maybe seven
  \end{itemize}
  \end{pptWide}}

\plush{\pptChapter{Methods}}

\begin{lnSnippet}[purpose-utils.java]
class Circle {
  public float radius;
}

class GeometryUtils {
  static float calcArea(Circle c) {
    return c.radius * c.radius * 3.14;
  }
}
\end{lnSnippet}
\begin{lnSnippet}[purpose-oop.java]
class Circle {
  public float radius;
  float area() {
    return radius * radius * 3.14;
  }
}
\end{lnSnippet}
\lnPitch{
  \pptSection[Purpose]{What Static Methods Are For?}
  \begin{pptWide}{2}
  {\small\ffinput{purpose-utils.java}}
  \par\columnbreak\par
  {\small\ffinput{purpose-oop.java}}
  \end{pptWide}
  \par
  Most notable Java examples~\citep{bugayenko2015blog0220}:
  FileUtils, IOUtils, and StringUtils from
  \href{https://commons.apache.org/}{Apache Commons};
  Files from \href{https://openjdk.org/projects/jdk7/}{JDK~7};
  Iterators from \href{https://github.com/google/guava}{Google Guava}.}

\lnQuote
  {david-west}
  {Traditional computer thinking treats data as passive and helpless, manipulated and moved by active procedures. Procedures tend to be egocentric and think that all data, its definitions and its values, are properly determined by the procedure currently using that data. The result is data that is inconsistent in definition, inconsistent in value, inaccessible because some other procedure has possession, or needlessly duplicated.}
  {west2004object}

\print{\pptSection[Problems]{What's Wrong with ``Utils''?}}
\print{1) They are \ul{unbreakable} dependencies}
\print{2) They are \ul{eager}, not lazy}
\plush{3) They are \ul{not cohesive}}

\begin{lnSnippet}[coupling-utils.java]
void paintIt(Circle c) {
  float s = GeometryUtils.calcArea(c);
  float p = s * 5.55;
  // paint it using the "p"
}
\end{lnSnippet}
\begin{lnSnippet}[coupling-oop.java]
void paintIt(Circle c) {
  float s = c.area();
  float p = s * 5.55;
  // paint it using the "p"
}
\end{lnSnippet}
\lnPitch{
  \pptSection[Coupling]{1) Tight Coupling}
  \begin{pptWide}{2}
  {\small\ffinput{coupling-utils.java}}
  \par\columnbreak\par
  {\small\ffinput{coupling-oop.java}}
  \end{pptWide}
  \par
  Which snippet is easier to test? Try to write a test for the first one,
  expecting \ff{s} to be equal to \ff{42.0}~\citep{bugayenko2014blog0505}.}

\begin{lnSnippet}[eagerness-imperative.java]
void paintIt(Circle c) {
  float s = GeometryUtils.calcArea(c);
  if (t) { return; }
  float p = s * 5.55;
  // paint it using the "p"
}
\end{lnSnippet}
\begin{lnSnippet}[eagerness-declarative.java]
void paintIt(Circle c) {
  float s = new AreaOf(c);
  if (t) { return; }
  float p = s * 5.55;
  // paint it using the "p"
}
\end{lnSnippet}
\lnPitch{
  \pptSection[Eagerness]{2) Imperative, not Declarative}
  \begin{pptWide}{2}
  {\small\ffinput{eagerness-imperative.java}}
  \par\columnbreak\par
  {\small\ffinput{eagerness-declarative.java}}
  \end{pptWide}
  \par
  Which snippet is more eager to calculate the area of the circle?
  Which one does it when it's \ul{really} necessary?~\citep{bugayenko2015blog0226}}

\begin{lnSnippet}[cohesion-utils.java]
class GeometryUtils {
  static float calcArea(Circle c);
  static float calcPerimeter(Circle c);
  static float calcSinus(Angle a);
  static float calcCosinus(float s);
  // and many more...
}
\end{lnSnippet}
\begin{lnSnippet}[cohesion-oop.java]
class Circle {
  float area();
  float perimeter();
}
class Angle {
  float sinus();
}
class Float {
  float cosinus();
}
\end{lnSnippet}
\lnPitch{
  \pptSection[Cohesion]{3) Low Cohesion}
  \begin{pptWide}{2}
  {\small\ffinput{cohesion-utils.java}}
  \par\columnbreak\par
  {\small\ffinput{cohesion-oop.java}}
  \end{pptWide}
  \par
  Which class looks more cohesive to you, the utility class \ff{GeometryUtils} or the \ff{Circle}?}

\plush{\pptChapter{Attributes}}

\begin{lnSnippet}[literals-constants.java]
class Constants {
  public static float PI = 3.1415926;
  public static String UTF_8 = "utf-8";
  public static String LOCALE = "fr";
  // and many more
}

println("S'il vous pla(*@\^{i}@*)t",
  Constants.LOCALE);
printf("It is %see speech!",
  Constants.LOCALE);
\end{lnSnippet}
\begin{lnSnippet}[literals-oop.java]
class Print { }
class TextInFrench { }

new Print(
  new TextInFrench(
    "S'il vous pla(*@\^{i}@*)t"
  )
)
\end{lnSnippet}
\lnPitch{
  \pptSection[Literals]{Public literals}
  \begin{pptWide}{2}
  {\small\ffinput{literals-constants.java}}
  \par\columnbreak\par
  {\small\ffinput{literals-oop.java}}
  \end{pptWide}
  \par
  We must solve the problem of functionality duplication, not just data duplication~\citep{bugayenko2015blog0706}.}

\begin{lnSnippet}[singletons-singleton.java]
class Canvas {
  public static Canvas INSTANCE =
    new Canvas();
  private Canvas() {}
  public void addCircle(Circle c);
}

Canvas.INSTANCE.addCircle(c1);
Canvas.INSTANCE.addCircle(c2);
\end{lnSnippet}
\begin{lnSnippet}[singletons-oop.java]
c = new Canvas();
c.addCircle(c1);
c.addCircle(c2);
\end{lnSnippet}
\lnPitch{
  \pptSection[Singletons]{Singletons}
  \begin{pptWide}{2}
  {\small\ffinput{singletons-singleton.java}}
  \par\columnbreak\par
  {\small\ffinput{singletons-oop.java}}
  \end{pptWide}
  \par
  Forget about singletons; never use them. Turn them into dependencies and pass
  them from object to object through the operator \ff{new}~\citep{bugayenko2016blog0627}.}

\plush{\pptChapter[FastJson]{FastJson as Example}}

\pptSection{JSONPath}
\begin{pptWide}{2}
\pptPic{.8}{size1.png}\par
\pptPic{.8}{size2.png}\par
\par\columnbreak\par
Some mind-blowing statistics:
\begin{itemize}
  \item 3 constructors
  \item 1 interface implemented
  \item 59 methods at \ff{JSONPath}
  \item 4,363 lines of code
  \item 34 inner static classes
  \item 74 times \ff{static} keyword
\end{itemize}
\end{pptWide}
\par
{\tiny \url{https://github.com/alibaba/fastjson/blob/master/src/main/java/com/alibaba/fastjson/JSONPath.java}\par}
\plush{}


\end{document}

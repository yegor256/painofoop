% (The MIT License)
%
% Copyright (c) 2023-2024 Yegor Bugayenko
%
% Permission is hereby granted, free of charge, to any person obtaining a copy
% of this software and associated documentation files (the 'Software'), to deal
% in the Software without restriction, including without limitation the rights
% to use, copy, modify, merge, publish, distribute, sublicense, and/or sell
% copies of the Software, and to permit persons to whom the Software is
% furnished to do so, subject to the following conditions:
%
% The above copyright notice and this permission notice shall be included in all
% copies or substantial portions of the Software.
%
% THE SOFTWARE IS PROVIDED 'AS IS', WITHOUT WARRANTY OF ANY KIND, EXPRESS OR
% IMPLIED, INCLUDING BUT NOT LIMITED TO THE WARRANTIES OF MERCHANTABILITY,
% FITNESS FOR A PARTICULAR PURPOSE AND NONINFRINGEMENT. IN NO EVENT SHALL THE
% AUTHORS OR COPYRIGHT HOLDERS BE LIABLE FOR ANY CLAIM, DAMAGES OR OTHER
% LIABILITY, WHETHER IN AN ACTION OF CONTRACT, TORT OR OTHERWISE, ARISING FROM,
% OUT OF OR IN CONNECTION WITH THE SOFTWARE OR THE USE OR OTHER DEALINGS IN THE
% SOFTWARE.

\documentclass[nobrand,anonymous,nodate,nosecurity]{huawei}
\usepackage{enumerate}
\usepackage{multicol}
\usepackage{href-ul}
\usepackage[noframes]{ffcode}
\usepackage{soul}
\usepackage{svg}
\begin{document}

\includesvg[height=5em]{cactus.svg}

\section*{Pain of OOP}

This series of lectures by \href{https://www.yegor256.com}{Yegor Bugayenko}
was first presented to students at
\href{https://innopolis.university/en/}{Innopolis University} in 2023
and \href{https://www.youtube.com/playlist?list=PLaIsQH4uc08ytf8POIIAkkR4ZsRq8DFiV}{video recorded}.
The complete set of slide decks is hosted and maintained in the
\href{https://github.com/yegor256/painofoop}{yegor256/painofoop}
GitHub repository.

\begin{abstract}
The course is a critical review of the current situation in object-oriented programming,
especially in the Java, C++, Ruby, and JavaScript worlds. In this course, certain programming
idioms, which are sometimes called ``best practices,'' are criticized for their
negative impact on code quality. These include static methods, NULL references, getters
and setters, ORM and DTO, annotations, traits and mixins, inheritance, and many others.
Instead, much ``cleaner'' object-oriented programming practices will be proposed.
\end{abstract}

% \section*{Introduction}

\textbf{What is the goal?}\\
The primary objective of the course is to help students understand the
difference between ``object thinking,'' originally motivated the
appearance of OOP, and the modern practices that often
severly impact the quality of code in a negative way.

\textbf{Who is the teacher?}\\
Yegor is developing software for more than 30 years, being a hands-on programmer
(see his GitHub account: \href{https://github.com/yegor256}{@yegor256})
and a manager of other programmers. At the moment Yegor is a director
of an R\&D laboratory in Huawei (Moscow, Russia). His primary research focus is
software quality problems. Some of the lectures he has recently presented
at some software conferences could be found at
\href{https://www.youtube.com/channel/UCr9qCdqXLm2SU0BIs6d_68Q}{his YouTube channel}.
Yegor also published a \href{https://www.yegor256.com/books.html}{few books}
and wrote a \href{https://www.yegor256.com/contents.html}{blog} about software engineering
and object-oriented programming.
Yegor previously tought a few courses in
Innopolis University (Kazan, Russia)
and HSE University (Moscow, Russia),
for example,
\href{https://github.com/yegor256/ppa}{Practical Program Analysis (2023)}
and
\href{https://github.com/yegor256/sqm}{Software Quality Metrics (2024)}
(all videos are available).

\textbf{Why this course?}\\
Maintainability of object-oriented software that most of us programmers write these days is
way below our expectations. One of the main reasons for that is our misunderstanding
of what objects are. This course may help clear things up.

\textbf{What's the methodology?}\\
Each lecture is a critical review of one of the existing ``best practices,'' such as
static methods or getters, with an intent to highlight its negative impact
on the quality of software.

\newpage
\section*{Course Structure}

Prerequisites to the course (it is expected that a student knows this):

\begin{itemize}
\item How to write code
\item How to design software
\item How to use Git and GitHub
\end{itemize}

After the course a student \ul{hopefully} will understand:

\begin{itemize}
\item What is the difference between objects and data?
\item Why are static methods bad?
\item What is immutability and why is it good?
\item How should one design a constructor?
\item How can exceptions be handled correctly?
\item What is data hiding for?
\item What's wrong with Printers, Writers, Scanners, and Readers?
\item Why are NULL references considered a billion-dollar mistake?
\item What's wrong with mixins and traits?
\item How can getters and setters be avoided?
\item How is declarative programming better than imperative programming?
\item Why is composition better than inheritance?
\item Where should data be stored if DTOs are considered bad practice?
\item How can the ORM design pattern be avoided?
\item Why are long variable names considered bad design?
\item Why are type casting and type checking against OOP principles?
\item What is cohesion and why does it matter?
\item How can SOLID and SRP principles be applied?
\item What is Inversion of Control for, and why are DI Containers considered harmful?
\item Why is MVC considered a bad design idea?
\end{itemize}

\newpage
\section*{Lectures}

The following topics are discussed:

\newlist{lectures}{enumerate}{10}
\setlist[lectures]{label*=\arabic*.}
\begin{lectures}
\item Algorithms
\item Static Methods
\item Getters
\item Setters and Mutability
\item ``-ER'' suffix
\item NULL references
\item Type casting and reflection
\item Inheritance
\item Annotations
\item Globals and DI Containers
\end{lectures}

\newpage
\section*{Grading}

In order to pass the course, students must attend lectures, labs, and
contribute to one of the following GitHub repositories, which are written
in more or less ``pure'' object-oriented style:

\begin{itemize}
    \item \href{https://github.com/yegor256/cactoos}{\ff{yegor256/cactoos}} (Java + Maven)
    \item \href{https://github.com/yegor256/takes}{\ff{yegor256/takes}} (Java + XSLT)
    \item \href{https://github.com/objectionary/eo2js}{\ff{objectionary/eo2js}} (JavaScript)
    \item \href{https://github.com/zerocracy/baza}{\ff{zerocracy/baza}} (Ruby + Sinatra + PostgreSQL)
    \item \href{https://github.com/yegor256/cam}{\ff{yegor256/cam}} (Bash + Python + Make)
\end{itemize}

There is no exam at the end of the course. Instead,
each student earns points for the following results:\\
\renewcommand{\arraystretch}{1}
\begin{tabular}{lrr}
Result & Points & Limit \\
\hline
Attended a lecture & +1 & 6 \\
Attended a lab & +1 & 8 \\
Submitted a ticket (that was accepted) & +2 & 8 \\
Submitted a pull request (that was merged) & +4 & 32 \\
\end{tabular}

Then, 25+ points mean ``A,'' 17+ mean ``B,'' and 9+ mean ``C.''

An online lecture is counted as ``attended'' only if a student was personally
presented in Zoom for more than 75\% of the lecture's time. Watching the
lecture from the computer of a friend \ul{doesn't} count.

\newpage
\section*{Learning Material}

The following books are highly recommended to read (in no particular order):

\begin{multicols}{2}\small\raggedright
{David West}, \ul{Object Thinking}, 2004\\[3pt]
{David West}, \ul{Design Thinking}, 2017\\[3pt]
{Robert Martin}, \ul{Clean Code}, 2008\\[3pt]
{Steve McConnell}, \ul{Code Complete}, 1993\\[3pt]
{Yegor Bugayenko}, \ul{Elegant Objects}, 2016\\[3pt]
Blog posts of Yegor Bugayenko, \href{https://www.yegor256.com/tag/oop}{on his blog}\\[3pt]
Video lectures of Yegor Bugayenko \href{https://www.youtube.com/playlist?list=PLaIsQH4uc08yw2CsNv5OV30GfKE6XVGii}{on YouTube}\\[3pt]
Object Thinking meetup presentations, \href{https://www.youtube.com/watch?v=yT6oO28wEik&list=PLaIsQH4uc08yetzX86w1pPck1QtGEy_ik}{on YouTube}\\[3pt]
\end{multicols}

\end{document}
